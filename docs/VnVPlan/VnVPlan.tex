\documentclass[12pt, titlepage]{article}

\usepackage{booktabs}
\usepackage{tabularx}
\usepackage{hyperref}
\hypersetup{
    colorlinks,
    citecolor=blue,
    filecolor=black,
    linkcolor=red,
    urlcolor=blue
}
\usepackage[round]{natbib}
\usepackage{amssymb}
%% Comments

\usepackage{color}

\newif\ifcomments\commentstrue

\ifcomments
\newcommand{\authornote}[3]{\textcolor{#1}{[#3 ---#2]}}
\newcommand{\todo}[1]{\textcolor{red}{[TODO: #1]}}
\else
\newcommand{\authornote}[3]{}
\newcommand{\todo}[1]{}
\fi

\newcommand{\wss}[1]{\authornote{blue}{SS}{#1}} 
\newcommand{\plt}[1]{\authornote{magenta}{TPLT}{#1}} %For explanation of the template
\newcommand{\an}[1]{\authornote{cyan}{Author}{#1}}

%% Common Parts

\newcommand{\progname}{SPDFM} % PUT YOUR PROGRAM NAME HERE %Every program
                                % should have a name


\begin{document}

\title{Project Title: System Verification and Validation Plan for \progname{}} 
\author{S. Shayan Mousavi M.}
\date{\today}
	
\maketitle

\pagenumbering{roman}

\section{Revision History}

\begin{tabularx}{\textwidth}{p{3cm}p{2cm}X}
\toprule {\bf Date} & {\bf Version} & {\bf Notes}\\
\midrule
Oct 17 2020& 1.0 & First Draft\\
\bottomrule
\end{tabularx}

\newpage

\tableofcontents

\listoftables

\listoffigures

\newpage

\section{Symbols, Abbreviations and Acronyms}

\renewcommand{\arraystretch}{1.2}
\begin{tabular}{l l} 
  \toprule		
  \textbf{symbol} & \textbf{description}\\
  \midrule 
  T & Test\\
  VnV & verification and validation\\
  \progname{} & Surface Plasmon Dynamics Finite Method\\
  \bottomrule
\end{tabular}\\
The complete table of symbols, abbreviations and acronyms can be found in the \href{https://github.com/shmouses/SPDFM/tree/master/docs/SRS}{SRS} document of the software. 
\newpage

\pagenumbering{arabic}

This document provides the information on validation and verification plans implemented for the \progname{} software. In this regard,the general approaches and plans are initially discussed and afterwards specific test cases and approaches for validation and verification of functional and nonfunctional requirements (can be found in \href{https://github.com/shmouses/SPDFM/tree/master/docs/SRS}{SRS}) are reviewed. VnV plans here are a combination of manual (assigned to a member of the VnV team to assess) and automated testing approaches to evaluate the correctness of the information (whether input or output) or satisfaction of a goal in \progname{}.    

\section{General Information}

\subsection{Summary}

The \progname{} software, which its VnV plan is discussed in this document, is a software for calculating plasmon-enhanced electric field and electric current in a meshed geometry. This software should be able to setup an optical source (given the related parameters) and study how electric field and current densities in the dielectric are affected. The calculations in this software are based on the newly established theory of surface plasmon oscillations, nonlocal hydrodynamic theory of surface plasmons \cite{hiremath2012numerical}.   

\subsection{Objectives}
\label{obj}

This document tries to address the most important areas of the \progname{} software that can act like bottle necks of the system and make sure these areas function properly. In this regard, following chapters will discuss how these key aspects which include, setting up a light source, having properly meshed geometry, well defined dielectric environment, and theoretical formulation used. These areas are reflected within the functional and nonfunctional system requirements in Chapter 5 of the \href{https://github.com/shmouses/SPDFM/tree/master/docs/SRS}{SRS} document.  
In this software some python libraries are implemented such as FEniCS and Meshio which their validity is accepted will not be checked here. 

\subsection{Relevant Documentation}

The relevant documentation for \progname{}, including the problem statement (\cite{SPDFMdoc22:online}), SRS (\cite{SPDFMdoc90:online}), SRS checklist (\cite{BlankPro77:online}), VnV report, MG, and MIS can be found in the devoted \href{https://github.com/shmouses/SPDFM}{GitHub repository} to this software. For theoretical aspects \cite{hiremath2012numerical}, \cite{monk2003finite}, and \cite{maier2007plasmonics} are the major important resources used in this software.  


\section{Plan}
	
\subsection{Verification and Validation Team}

The VnV team in this work includes S. Shayan Mousavi M., responsible for reviewing all the documentations, providing test cases and their execution, verifying the theoretical aspects and their implementation; Dr. Spencer Smiths (CAS741 instructor), and S. Parsa Tayefeh Morsal (domain expert), responsible for reviewing the design of the software, all documentations, and the documenting style itself; Naveen Ganesh Muralidharan (secondary reviewer) responsible for reviewing SRS document; Siddharth (Sid) Shinde (secondary reviewer) responsible for reviewing VnV document; Gabriela Sánchez Díaz (secondary reviewer) responsible for reviewing MG and MIS documents. The theoretical aspects and finite element method implemented in \progname{} is verified by Dr. Gianluigi Botton (supervisor), and Dr. Alexander Pofelski (field expert contributor).



\subsection{SRS Verification Plan}

The \progname{} shall be verified in the following ways:\\
\\
 Initial reviews from assigned members of the VnV team (Dr. Spencer Smith, S. Parsa Tayefeh Morsal, Naveen Ganesh Muralidharan, and Shayan Mousavi). In this regard, the document shall be manually reviewed using the \href{https://gitlab.cas.mcmaster.ca/smiths/cas741/-/blob/master/BlankProjectTemplate/docs/SRS/SRS-Checklist.pdf}{SRS checklist} (\cite{BlankPro77:online}) upon its initial version.\\
 \\
 Secondary review by the author (Shayan Mousavi). The SRS document shall be reviewed after receiving initial reviews, and completion of \href{https://github.com/shmouses/SPDFM/tree/master/docs/VnV}{VnV} document.\\  
 \\
 Final review by the author (Shayan Mousavi) and the instructor (Dr. Spencer Smith). The document shall be manually reviewed according to the  \href{https://gitlab.cas.mcmaster.ca/smiths/cas741/-/blob/master/BlankProjectTemplate/docs/SRS/SRS-Checklist.pdf}{SRS checklist} (\cite{BlankPro77:online})  after MG and MIS development.\\ 
 \\
 Review of theoretical aspects by the field experts (Dr. Gianluigi Botton, Dr. Alexander Pofelski, and Shayan Mousavi). Theories used in the SRS document shall be reviewed manually with respect to the governing relations in the realm of plasmonic physics (\cite{maier2007plasmonics}, \cite{hiremath2012numerical}, \cite{monk2003finite}).\\
 \\ 
 Feedback received from interested contributors through issue tracker in GitHub platform will also be used to improve this document. 
 

\subsection{Design Verification Plan}

The design shall be verified by ensuring that key aspects in \progname{} are, as listed in Section \ref{obj}. In this regard, the system functional requirements shall be tested initially, as outlined in \ref{func}, and in following the nonfunctional requirements will be review, as outlined in \ref{nonfunc}.

\subsection{Implementation Verification Plan}

The implementation shall be verified in the following ways:\\
\\
Code Walkthrough by author:  Module unit code shall be inspected for functional errors by the author (Shayan Mousavi) after MIS document is developed.\\  
\\
Code Walkthrough by contributors: Module unit code shall be inspected for functional errors by Dr. Alexander Pofelski after MIS document is developed.\\
\\
System Tests:  System tests will be carried out as listed in Section \ref{systest}. The functional and nonfunctional requirements shall be by those listed in the outline of each test. These tests are conducted either manually or automatically which is outlined for each test individual in Chapter \ref{systest}. 

\subsection{Automated Testing and Verification Tools}

By predetermining some user inputs, an automated testing approach can be
considered. The most of the functional requirement system tests will be executed and
validated automatically; these tests are outlined in Section \ref{func}. The goal of testing is $100\%$ code coverage. However, some system tests () require at least a manual assessment of the test results and
are thus removed from the automated scope.
\subsection{Software Validation Plan}
 
Metallic Nanoparticle Boundary Element Method (MNPBEM) toolbox in MATLAB is used for validation of the calculated electric field density in this study. In this regard, as MNPBEM toolbox uses quasi-static method to solve Maxwell's equations at the boundaries of a meshed geometry, this software is used to validation of the \progname{}, which solves similar equations using nonlocal hydrodynamic theory (Test 18).

\section{System Test Description}
\label{systest}

\subsection{Tests for Functional Requirements}
\label{func}
The subsections below are designed to cover all the functional requirements of the system which are listed in Chapter 5 of the \href{https://github.com/shmouses/SPDFM/tree/master/docs/SRS}{SRS} document. coverage of each functional requirement is indicated in each subsection.

\subsubsection{Input Verification (and/or Validation) tests}


This section covers verification and validation of the user's inputs which are reflected in R2, R4, R5.
		
\paragraph{Test R 2: Correctness of the light source related input data}
\begin{enumerate}		

%----------------------------------------------------------------T1
\item{\textbf{Test 1:} validity of \textbf{p} vector\\}

Control: Automatic
					
Initial State: N/A
					
Input: Incident light polarity: \textbf{p} = (a, 0, 0)\\
\rule{1.15cm}{0pt} Direction: \textbf{d} = (0, 1, 0)
					
Output: An error should be print out saying "Invalid parameter p: p should be a 3D vector of floating numbers"  

Test Case Derivation: Physically \textbf{p} and \textbf{d} must be 3D vectors in $\Re^3$ which are perpendicular to each other.
 				
How test will be performed: Using Pytest library in python Input Module will be fed with the input, \textbf{p} and \textbf{d} vectors, and the output will be checked (\href{https://github.com/shmouses/SPDFM/tree/master/src}{p\_d\_test.py}). Shayan Mousavi is responsible for execution of this test. 
\\
\\
%----------------------------------------------------------------T2
%
\item{\textbf{Test 2:} validity of \textbf{d} vector \\}

Control: Automatic

Initial State: N/A

Input: Incident light polarity: \textbf{p} = (1, 0, 0)\\
\rule{1.15cm}{0pt} Direction: \textbf{d} = (0, 0, x)

Output: An error should be print out saying "Invalid parameter d: d should be a 3D vector of floating numbers"  

Test Case Derivation: Physically \textbf{p} and \textbf{d} must be 3D unit vector in $\Re^3$ which are perpendicular to each other.

How test will be performed: Using Pytest library in python Input Module will be fed with the input, \textbf{p} and \textbf{d} vectors, and the output will be checked (\href{https://github.com/shmouses/SPDFM/tree/master/src}{p\_d\_test.py}). Shayan Mousavi is responsible for execution of this test. 
\\
\\
%----------------------------------------------------------------T3

\item{\textbf{Test 3:} validation of \textbf{d} vector unity \\}

Control: Automatic

Initial State: N/A

Input: Incident light polarity: \textbf{p} = (1, 0, 0)\\
\rule{1.15cm}{0pt} Direction: \textbf{d} = (0, 1, 0.2)

Output: An error should be print out saying "Invalid parameter d: d should be a 3D unit vector of floating numbers"  

Test Case Derivation: Physically \textbf{p} and \textbf{d} must be 3D vectors in $\Re^3$ which are perpendicular to each other.

How test will be performed: Using Pytest library in python Input Module will be fed with the input, \textbf{p} and \textbf{d} vectors, and the output will be checked (\href{https://github.com/shmouses/SPDFM/tree/master/src}{p\_d\_test.py}). Shayan Mousavi is responsible for execution of this test.
\\
\\
%----------------------------------------------------------------T4

\item{\textbf{Test 4:} validation of \textbf{p} and \textbf{d} vectors orthogonality \\}

Control: Automatic

Initial State: N/A

Input: Incident light polarity: \textbf{p} = (1, 0, 0)\\
\rule{1.15cm}{0pt} Direction: \textbf{d} = (1, 0, 0)

Output: An error should be print out saying "Invalid parameters p and d: orthogonality of p and d vectors is not followed"  

Test Case Derivation: Physically \textbf{p} and \textbf{d} must be 3D vectors in $\Re^3$ which are perpendicular to each other.

How test will be performed: Using Pytest library in python Input Module will be fed with the input, \textbf{p} and \textbf{d} vectors, and the output will be checked (\href{https://github.com/shmouses/SPDFM/tree/master/src}{p\_d\_test.py}). Shayan Mousavi is responsible for execution of this test.
\\
\\
%----------------------------------------------------------------T5

\item{\textbf{Test 5:} validation of accepting correct p and d input\\}

Control: Automatic

Initial State: N/A

Input: Incident light polarity: \textbf{p} = (1, 0, 0)\\
\rule{1.15cm}{0pt} Direction: \textbf{d} = (0, 1, 0)

Output: This message should be appear: "Valid polarity and direction vectors" and software should move on

Test Case Derivation: User will be informed of correctness of the input.

How test will be performed: Using Pytest library in python Input Module will be fed with the input, \textbf{p} and \textbf{d} vectors, and the output will be checked (\href{https://github.com/shmouses/SPDFM/tree/master/src}{p\_d\_test.py}). Shayan Mousavi is responsible for execution of this test.
\\
\\
%----------------------------------------------------------------T6

\item{\textbf{Test 6:} validation of wavelength input\\}

Control: Automatic

Initial State: for having the code moving forward and accepting the wavelength value, an acceptable set of \textbf{p}, and \textbf{d} data should be already given to the software.\\ 
\rule{2.3cm}{0pt} \textbf{p} = (1,0,0)\\
\rule{2.3cm}{0pt} \textbf{d} = (0,1,0)

Input: wavelength = 100

Output: An error should be print out saying "Out of range wavelength input: Input wavelength should be in nm and between 200 nm to 6000 nm" 

Test Case Derivation: SPDFM only accepts wavelength values for the light source between 200 nm to 6000 nm as the surface plasmonic behaviour of most of the known materials is observed in this energy range (Table 1, Section 4.2.6, \href{https://github.com/shmouses/SPDFM/tree/master/doc/SRS}{SRS} document). 

How test will be performed: Using Pytest library in python Input Module will be fed with the inputs, \textbf{p}, \textbf{d}, and the wavelength value then the output will be checked (\href{https://github.com/shmouses/SPDFM/tree/master/src}{wavelength\_test.py}). Shayan Mousavi is responsible for execution of this test.
\\
\\
%----------------------------------------------------------------T7
\item{\textbf{Test 7:} validation of wavelength input\\}

Control: Automatic

Initial State: for having the code moving forward and accepting the wavelength value, an acceptable set of \textbf{p}, and \textbf{d} data should be already given to the software.\\ 
\rule{2.3cm}{0pt} \textbf{p} = (1,0,0)\\
\rule{2.3cm}{0pt} \textbf{d} = (0,1,0)

Input: wavelength = 6200
Output: An error should be print out saying "Out of range wavelength input: Input wavelength should be in nm and between 200 nm to 6000 nm" 

Test Case Derivation: SPDFM only accepts wavelength values for the light source between 200 nm to 6000 nm as the surface plasmonic behaviour of most of the known materials is observed in this energy range (Table 1, Section 4.2.6, \href{https://github.com/shmouses/SPDFM/tree/master/doc/SRS}{SRS} document). 

How test will be performed: Using Pytest library in python Input Module will be fed with the inputs, \textbf{p}, \textbf{d}, and the wavelength value then the output will be checked (\href{https://github.com/shmouses/SPDFM/tree/master/src}{wavelength\_test.py}). Shayan Mousavi is responsible for execution of this test.
\\
\\
%----------------------------------------------------------------T8
\item{\textbf{Test 8:} validation of wavelength input\\}

Control: Automatic

Initial State: for having the code moving forward and accepting the wavelength value, an acceptable set of \textbf{p}, and \textbf{d} data should be already given to the software.\\ 
\rule{2.3cm}{0pt} \textbf{p} = (1,0,0)\\
\rule{2.3cm}{0pt} \textbf{d} = (0,1,0)

Input: wavelength = 700

Output: A confirmation massage should be showed up: "The input wavelength is valid" and software must move on.

Test Case Derivation: SPDFM only accepts wavelength values for the light source between 200 nm to 6000 nm as the surface plasmonic behaviour of most of the known materials is observed in this energy range (Table 1, Section 4.2.6, \href{https://github.com/shmouses/SPDFM/tree/master/doc/SRS}{SRS} document). 

How test will be performed: Using Pytest library in python Input Module will be fed with the inputs, \textbf{p}, \textbf{d}, and the wavelength value then the output will be checked (\href{https://github.com/shmouses/SPDFM/tree/master/src}{wavelength\_test.py}). Shayan Mousavi is responsible for execution of this test. 
\\
\\
\end{enumerate}

\paragraph{Test R 4: Correctness of the materials properties data entry}
\begin{enumerate}
%---------------------------------------------------------------T9
\item{\textbf{Test 9:} \\}

Control: Manual

Initial State: As materials properties as assumed to be received by another input module no initial input regarding the light source is required. The validity of the material properties is not been tested and focus is on providing the complete data set.

Input: \href{https://github.com/shmouses/SPDFM/blob/master/src/TestCase/data_missing_set1.csv}{data\_missing\_set1.csv} 

All the material related parameters are written in a .csv file as instructed in the \href{https://github.com/shmouses/SPDFM/blob/master/src/TestCase/Input_Instruction.txt}{Input\_Instruction.txt} more details can also be found in MG and MIS documents. 

Output: An error saying "Error: Insufficient properties are provided"

Test Case Derivation: Although some constraints on material parameters can be considered, \progname{} has no limitations on these values. However for initiation of the simulations, \progname{} need to be given enough input data in a .csv file.  

How test will be performed: Using Pytest library the material properties which are written in \href{https://github.com/shmouses/SPDFM/blob/master/src/TestCase/data_missing_set1.csv}{data\_missing\_set1.csv} file will be input to Material Properties Input Module. As in this data set $\mu_0$ is missed, desired output is an error. Shayan Mousavi is responsible for execution of this test.  

 
\end{enumerate}

\paragraph{Test R 5: Mesh Input Verification}
\begin{enumerate}
%---------------------------------------------------------------T10	
\item{\textbf{Test 10:} validation of format of the input mesh\\}

Control: Automated

Initial State: As \progname{} solves the nonlocal hydrodynamic equations for any arbitrary meshed geometry the quality of the mesh has not been verified. The only area about mesh that is concerned by \progname{} is the input format of the mesh. 

Input:A mesh cylinder in a file with .geo format (\href{https://github.com/shmouses/SPDFM/blob/master/src/TestCase/cylinder\_3d.geo}{cylinder\_3d.geo}) 

Output: An error saying "Error: Invalid mesh file format" 

Test Case Derivation: \progname{} only accepts .mesh format as the mesh input. 

How test will be performed: The file with an unexpected format should be input to the software and the expected error shall be appear on the screen. Shayan Mousavi is responsible for execution of this test.  


%---------------------------------------------------------------T11	
\item{\textbf{Test 11:} validation of format of the input mesh\\}

Control: Automated

Initial State: As \progname{} solves the nonlocal hydrodynamic equations for any arbitrary meshed geometry the quality of the mesh has not been verified. The only area about mesh that is concerned by \progname{} is the input format of the mesh. 

Input:A mesh cylinder in a file with .mesh format (\href{https://github.com/shmouses/SPDFM/blob/master/src/TestCase/cylinder\_3d.mesh}{cylinder\_3d.mesh}) 

Output:"Mesh received"

Test Case Derivation: \progname{} only accepts .mesh format as the mesh input. If the proper format is fed to the software, an approval message will confirm that software will move on to the next step.

How test will be performed: The file with an unexpected format should be input to the software and expected message shall be appear on the screen. Shayan Mousavi is responsible for execution of this test.  	
	
\end{enumerate}

\subsubsection{Output Verification (and/or Validation) tests}
\paragraph{Test R 1: Software providing the user with information }

\begin{enumerate}
	
	\item{\textbf{Test 12:} Instruction print out\\}
	
	Control: Manual
	
	Initial State: N/A
	
	Input: N/A
	
	Output: 
	"To initiate SPDFM software simulations user need to provide the software with following information:\\
	p = (p1, p2, p3) : 3D light source polarity vector, \{p1, p2, p3\} in R;\\
	d = (d1, d2, d3) : 3D light source propagation direction unit vector, \{d1, d2, d3\} in R;\\
	p and d must be perpendicular;\\ 
	Wavelength = wl : light source spatial wavelength (nm), wl in R;\\
	Frequency = freq : light source angular frequency (THz), freq in R;\\
	Time step = delt\_t : time step (fs), delta\_t in R;\\
	Final time = t\_final : (fs), t\_final in R;\\
	provide materials properties in a .csv as instructed;\\
	Mesh file : address of the .gmsh file."
	
	Test Case Derivation: As soon as the software is executed, the above paragraph should be shown to notify the user of the 
	
	How test will be performed: Shayan Mousavi is responsible to check and verify functionality of this feature.   
	
	
\end{enumerate}



\paragraph{Test R 3: Correctness of the light source electric field calculation}

\begin{enumerate}
	
	\item{\textbf{Test 13:} calculation of the light source electric field \\}
	
	Control: Automated 
	
	Initial State:  \textbf{p}, \textbf{d}, and wavelength are previously given to the software. \\
	\rule{2.3cm}{0pt} \textbf{p}=(1,0,0)\\
	\rule{2.3cm}{0pt} \textbf{d}=(0,1,0)\\
	\rule{2.3cm}{0pt} wavelength=700 nm\\
	\rule{2.3cm}{0pt} The location vector, r, and time, t, will be given as below.  
	
	Input: R = $\{\textbf{r}=(0.1*r_x,0,0)| \forall r_x \in \aleph, r_x \in [0,10]\}$\\
	\rule{1.3cm}{0pt}T = $\{t| \forall t \in \aleph, t \in (0,10]\}$
	
	Output: For all the elements in sets R and T the output for the $\|E_i-E_{i}^t\|$ should be equal zero.
	 	
	Test Case Derivation: In this test $E_i$ is the electric field of the incident beam calculated by the software and $E_i^t$ is the electric field calculated in the test function following the equation $E_i^t = cos(k\  \textbf{d.} \textbf{r} - \omega t) - i \ sin(k\  \textbf{d.} \textbf{r} - \omega t)$   
	
	How test will be performed: Using Pytest library in python. Codes can be found in \textbf{E\_field\_test.py } in the src folder. Shayan Mousavi is responsible for execution of this test. 
	
	
\item{\textbf{Test 14:} Light source electric field evolution\\}

Control: Manual 

Initial State:  \textbf{p}, \textbf{d}, and wavelength are previously given to the software. \\
\rule{2.3cm}{0pt} \textbf{p}=(1,0,0)\\
\rule{2.3cm}{0pt} \textbf{d}=(0,1,0)\\
\rule{2.3cm}{0pt} wavelength=700 nm\\
\rule{2.3cm}{0pt} The location vector, r, and time, t, will be given as below.  

Input: R = $\{\textbf{r}=(0.1*r_x,0,0)| \forall r_x \in \aleph, r_x \in [0,10]\}$\\
\rule{1.3cm}{0pt}T = $\{t| \forall t \in \aleph, t \in (0,10]\}$

Output: for t=0, plot of evolution of electric field by R. for r=(0, 0, 0), plot of evolution of electric field in time.

Test Case Derivation: In this test $E_i = cos(k\  \textbf{d.} \textbf{r} - \omega t) - i \ sin(k\  \textbf{d.} \textbf{r} - \omega t)$ is plotted with respect to time and space and the examiner is responsible to see if the behaviour of the function is as expected or not.   

How test will be performed: Using Pytest library in python. Codes can be found in \textbf{E\_field\_plot\_test.py } in the src folder. Shayan Mousavi is responsible for execution of this test. 
	
\end{enumerate}

\paragraph{Test R 6: Correctness of the calculated electric field density (E) and electric current density (J$_\textbf{{HD}}$)}
	
\begin{enumerate}
	
\item{\textbf{Test 15:} Proper implementation of equations in code\\}

Control: Manual

Initial State:  performance of FEniCS toolbox has not been tested. 

Input:N/A

Output: N/A

Test Case Derivation: Walkthrough 

How test will be performed: Shayan Mousavi and Dr. Alexander Pofelski are responsible for walkthrough the whole codes after MG and MIS documents developed and verify the functionality of the code and validate that theoretical equations are properly fed to FEniCS PDE solver.

	
\item{\textbf{Test 16:} Amplitude of E and J$_{HD}$ in absence of light source\\}

Control: Manual 

Initial State:  performance of FEniCS toolbox has not been tested. 

Input:\textbf{p}=(0,0,0)\\
\rule{1.3cm}{0pt}\textbf{d}=(0,1,0)\\
\rule{1.3cm}{0pt}wavelength=700 nm\\ 
\rule{1.3cm}{0pt}T = $\{t| \forall t \in \aleph, t \in (0,10]\}$\\
\rule{1.3cm}{0pt}materials properties file (.csv): \href{https://github.com/shmouses/SPDFM/blob/master/src/TestCase/complete\_set.csv}{complete\_set.csv}\\
\rule{1.3cm}{0pt}mesh file (.msh): \href{https://github.com/shmouses/SPDFM/blob/master/src/TestCase/cylinder\_3d.msh}{cylinder\_3d.msh}

Output: for all the nodes value of E and J$_{HD}$ shall be zero 

Test Case Derivation: As polarity of the incident light is p=(0,0,0), in fact, the amplitude of the incident electric field is zero. Thus, in absence of the excitation source no plasmonic activity should be observed in the medium. 

How test will be performed: Shayan Mousavi is responsible for inputting the data an inspecting if E and J$_{HD}$ values are equal zero.  	
		
		
\item{\textbf{Test 17:} Amplitude of E and J$_{HD}$ in absence of light source\\}

Control: Manual 

Initial State: Performance of FEniCS toolbox in python has not been tested. 

Input:\textbf{p}=(0,0,0)\\
\rule{1.3cm}{0pt}\textbf{d}=(0,1,0)\\
\rule{1.3cm}{0pt}wavelength=700 nm\\ 
\rule{1.3cm}{0pt}T = $\{t| \forall t \in \aleph, t \in (0,10]\}$\\
\rule{1.3cm}{0pt}materials properties file (.csv): Material properties for gold (\href{https://github.com/shmouses/SPDFM/blob/master/src/TestCase/complete\_set.csv}{complete\_set.csv})\\
\rule{1.3cm}{0pt}mesh file (.msh): A 3d meshed cylinder (\href{https://github.com/shmouses/SPDFM/blob/master/src/TestCase/cylinder\_3d.msh}{cylinder\_3d.msh})

Output: for all the nodes value of E and J$_{HD}$ shall be zero 

Test Case Derivation: As polarity of the incident light is p=(0,0,0), in fact, the amplitude of the incident electric field is zero. Thus, in absence of the excitation source no plasmonic activity should be observed in the medium. 

How test will be performed: Shayan Mousavi is responsible for inputting the data an inspecting if E and J$_{HD}$ values are equal zero.  


\item{\textbf{Test 18:} Amplitude of E and J$_{HD}$ in absence of light source\\}

Control: Manual 

Initial State:  performance of FEniCS toolbox in python and MNPBEM toobox in MATLAB has not been tested. 

Input:\textbf{p}=(1,0,0)\\
\rule{1.1cm}{0pt}\textbf{d}=(0,1,0)\\
\rule{1.1cm}{0pt}wavelength=532 nm\\ 
\rule{1.1cm}{0pt}t=0\\
\rule{1.1cm}{0pt}materials properties file (.csv): Material properties for gold  \href{https://github.com/shmouses/SPDFM/blob/master/src/TestCase/complete\_set.csv}{complete\_set.csv}\\
\rule{1.1cm}{0pt}mesh file (.msh): A meshed sphere of 12 nm diameter

Output: .vtk 3D intensity map of the meshed geometry

Test Case Derivation: Similar input shall be fed to the MNPBEM toolbox in  MATLAB and the electric field enhancement be extracted for the 3D meshed geometry. As MNPBEM uses boundary element method to calculate a quasi-static approximation of the Maxwell's equations it is not expected to have exactly the same results. However, as both nonlocal hydrodynamic theory and quasi-static theory trying to solve similar equations roughly similar behaviour on the surface of the structure is expected.     

How test will be performed: Shayan Mousavi is responsible for executing both SPDFM and MNPBEM simulations. Shayan Mousavi, Dr. Gianluigi Botton, and Dr. Alex Pofelski are responsible for evaluating how these two responses are close.


\end{enumerate}
	

\subsection{Tests for Nonfunctional Requirements}
\label{nonfunc}

\subsubsection{Usability}
		
\paragraph{Test NR1: Capability of execution of the software}

\begin{enumerate}

\item{\textbf{Test 19:} Usibility \\}

Type: Usability Survey
					
Initial State: The system being used should already have Python3, and FEniCS toolbox installed on.  
					
Input/Condition: A usability survey with the questions listed in Section \ref{usab}. For execution of a simulation, data provided in Test? is suggested to be used.
					
Output/Result: Survey results
					
How test will be performed: each participant shall install the software on a system and try to run a simulation. Respondents will be asked to rank their experience of installing and running a module. A final average grade of 3 will indicate that the users found the system to have average usability. The higher the score,the better the perception of usability. Shayan Mousavi and Alexander Pofelski shall be participate this test. This approach is suggested by \cite{LatticeB41:online}.

\end{enumerate}
\subsubsection{Maintainability}

\paragraph{Test NR2: Maintainability and expandability of the software}

\begin{enumerate}
	
	\item{\textbf{Test 20:} Maintainability\\}
	
	Type: Maintainability Walkthrough
	
	Initial State: Maintainability of the repository and external toolboxes used in this software such as FEniCS has not been tested.
	
	Input/Condition: production version of \progname{} has been released.
	
	Output/Result: A graded report describing the maintainability of the repository
	
	How test will be performed: Dr. Alexander Pofelski shall check the repository for the following documentation:  SRS, VnV Plan, MG, MIS, User Guide. He shall mark 1 point for each of the above documents. He shall read through each of the above documents and
	provide a grade between 1 and 5 for clarity of the writing. A score of 1 represents a document that is hard to understand, and a score of 5 represents a document that is easy to understand. The user should also grade the traceability of each document. A score of 1 represents no links within the report, and a score of 5 represents many links between sections of the report. The user shall then divide the sum of the scores for all of the reports by 5.
	 A final average grade of 3 will indicate that the users found the system to have average Maintainability. The higher the score,the better the perception of Maintainability. This approach is suggested by \cite{LatticeB41:online}. 
					
\end{enumerate}

\subsection{Traceability Between Test Cases and Requirements}

Table \ref{Table:A_trace} shows the connection between functional and nonfunctional requirements and the tests provided in this document. 

\begin{table}[h!]
	\centering
	\begin{tabular}{|c|c|c|c|c|c|c|c|c|c|c|c|c|c|c|c|c|c|c|c|c|}
		\hline        
		& R1 & R2 & R3 & R4 & R5 & R6 & NR1 & NR2 \\
		\hline
		Test1        & &X & & & & & &   \\ \hline
		Test2        & &X & & & & & &   \\ \hline
		Test3        & &X & & & & & &  \\ \hline
		Test4        & &X & & & & & & \\ \hline
		Test5        & &X & & & & & &   \\ \hline
		Test6        & &X & & & & & &  \\ \hline
		Test7        & &X & & & & & &  \\ \hline
		Test8        & &X & & & & & &  \\ \hline
		Test9        & & & &X & & & &  \\ \hline
		Test10       & & & & &X & & &  \\ \hline
		Test11       & & & & & X& & &  \\ \hline
		Test12       &X & & & & & & &  \\ \hline
		Test13       & & &X & & & & &  \\ \hline
		Test14       & & &X & & & & &  \\ \hline
		Test15       & & & & & &X & &  \\ \hline
		Test16       & & & & & &X & &  \\ \hline
		Test17       & & & & & &X & &  \\ \hline
		Test18       & & & & & &X & &  \\ \hline
		Test19       & & & & & & &X &  \\ \hline
		Test20       & & & & & & & &X  \\ \hline
	\end{tabular}
\caption{Traceability Matrix Showing the Connections Between Tests and Functional and Nonfunctional System Requirements}
\label{Table:A_trace}
\end{table}

\section{Unit Test Description}

\wss{Reference your MIS and explain your overall philosophy for test case
  selection.}  
\wss{This section should not be filled in until after the MIS has
  been completed.}

\subsection{Unit Testing Scope}

\wss{What modules are outside of the scope.  If there are modules that are
  developed by someone else, then you would say here if you aren't planning on
  verifying them.  There may also be modules that are part of your software, but
  have a lower priority for verification than others.  If this is the case,
  explain your rationale for the ranking of module importance.}

\subsection{Tests for Functional Requirements}

\wss{Most of the verification will be through automated unit testing.  If
  appropriate specific modules can be verified by a non-testing based
  technique.  That can also be documented in this section.}

\subsubsection{Module 1}

\wss{Include a blurb here to explain why the subsections below cover the module.
  References to the MIS would be good.  You will want tests from a black box
  perspective and from a white box perspective.  Explain to the reader how the
  tests were selected.}

\begin{enumerate}

\item{test-id1\\}

Type: \wss{Functional, Dynamic, Manual, Automatic, Static etc. Most will
  be automatic}
					
Initial State: 
					
Input: 
					
Output: \wss{The expected result for the given inputs}

Test Case Derivation: \wss{Justify the expected value given in the Output field}

How test will be performed: 
					
\item{test-id2\\}

Type: \wss{Functional, Dynamic, Manual, Automatic, Static etc. Most will
  be automatic}
					
Initial State: 
					
Input: 
					
Output: \wss{The expected result for the given inputs}

Test Case Derivation: \wss{Justify the expected value given in the Output field}

How test will be performed: 

\item{...\\}
    
\end{enumerate}

\subsubsection{Module 2}

...

\subsection{Tests for Nonfunctional Requirements}

\wss{If there is a module that needs to be independently assessed for
  performance, those test cases can go here.  In some projects, planning for
  nonfunctional tests of units will not be that relevant.}

\wss{These tests may involve collecting performance data from previously
  mentioned functional tests.}

\subsubsection{Module ?}
		
\begin{enumerate}

\item{test-id1\\}

Type: \wss{Functional, Dynamic, Manual, Automatic, Static etc. Most will
  be automatic}
					
Initial State: 
					
Input/Condition: 
					
Output/Result: 
					
How test will be performed: 
					
\item{test-id2\\}

Type: Functional, Dynamic, Manual, Static etc.
					
Initial State: 
					
Input: 
					
Output: 
					
How test will be performed: 

\end{enumerate}

\subsubsection{Module ?}

...

\subsection{Traceability Between Test Cases and Modules}

\wss{Provide evidence that all of the modules have been considered.}
				
\bibliographystyle{plainnat}

\bibliography{../../refs/References}

\newpage

\section{Appendix}

This is where you can place additional information.

\subsection{Symbolic Parameters}

The definition of the test cases will call for SYMBOLIC\_CONSTANTS.
Their values are defined in this section for easy maintenance.

\subsection{Usability Survey Questions?} \label{usab}

Using the following rubric please rate the five statements found below (this rubric is suggested by \cite{LatticeB41:online}):\\
\textbf{1. The formatting of the input file was easy to understand.} \\
\rule{2cm}{0pt}1 - strongly disagree\\
\rule{2cm}{0pt}2 - somewhat disagree\\
\rule{2cm}{0pt}3 - neither agree nor disagree\\
\rule{2cm}{0pt}4 - somewhat agree\\
\rule{2cm}{0pt}5 - strongly agree\\ 
\\
\textbf{2. The location to place the input file was easy to find.}\\
\rule{2cm}{0pt}1 - strongly disagree\\
\rule{2cm}{0pt}2 - somewhat disagree\\
\rule{2cm}{0pt}3 - neither agree nor disagree\\
\rule{2cm}{0pt}4 - somewhat agree\\
\rule{2cm}{0pt}5 - strongly agree\\ 
\\
\textbf{3. Navigating to the correct module was straightforward.}\\
\rule{2cm}{0pt}1 - strongly disagree\\
\rule{2cm}{0pt}2 - somewhat disagree\\
\rule{2cm}{0pt}3 - neither agree nor disagree\\
\rule{2cm}{0pt}4 - somewhat agree\\
\rule{2cm}{0pt}5 - strongly agree\\
\\
\textbf{4. The MG and MIS of this product explain the modules well.}\\
\rule{2cm}{0pt}1 - strongly disagree\\
\rule{2cm}{0pt}2 - somewhat disagree\\
\rule{2cm}{0pt}3 - neither agree nor disagree\\
\rule{2cm}{0pt}4 - somewhat agree\\
\rule{2cm}{0pt}5 - strongly agree\\
\\
\textbf{5. I would recommend this product.}\\
\rule{2cm}{0pt}1 - strongly disagree\\
\rule{2cm}{0pt}2 - somewhat disagree\\
\rule{2cm}{0pt}3 - neither agree nor disagree\\
\rule{2cm}{0pt}4 - somewhat agree\\
\rule{2cm}{0pt}5 - strongly agree\\
\end{document}