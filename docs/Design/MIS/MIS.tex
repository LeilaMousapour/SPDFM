\documentclass[12pt, titlepage]{article}

\usepackage{amsmath, mathtools}

\usepackage[round]{natbib} \usepackage{amsfonts} \usepackage{amssymb}
\usepackage{graphicx} \usepackage{colortbl} \usepackage{xr}
\usepackage{hyperref} \usepackage{longtable} \usepackage{xfrac}
\usepackage{tabularx} \usepackage{float} \usepackage{siunitx}
\usepackage{booktabs} \usepackage{multirow} \usepackage[section]{placeins}
\usepackage{caption} \usepackage{fullpage}

\hypersetup{ bookmarks=true,     % show bookmarks bar?
	colorlinks=true,       % false: boxed links; true: colored links
	linkcolor=red,          % color of internal links (change box color with
	% linkbordercolor)
	citecolor=blue,      % color of links to bibliography
	filecolor=magenta,  % color of file links
	urlcolor=cyan          % color of external links
}

\usepackage{array}

\externaldocument{../../SRS/SRS}

%% Comments

\usepackage{color}

\newif\ifcomments\commentstrue

\ifcomments
\newcommand{\authornote}[3]{\textcolor{#1}{[#3 ---#2]}}
\newcommand{\todo}[1]{\textcolor{red}{[TODO: #1]}}
\else
\newcommand{\authornote}[3]{}
\newcommand{\todo}[1]{}
\fi

\newcommand{\wss}[1]{\authornote{blue}{SS}{#1}} 
\newcommand{\plt}[1]{\authornote{magenta}{TPLT}{#1}} %For explanation of the template
\newcommand{\an}[1]{\authornote{cyan}{Author}{#1}}
 %% Common Parts

\newcommand{\progname}{SPDFM} % PUT YOUR PROGRAM NAME HERE %Every program
                                % should have a name


\begin{document}
	
	\title{Module Interface Specification for \progname{}}
	
	\author{S. Shayan Mousavi M.}
	
	\date{\today}
	
	\maketitle
	
	\pagenumbering{roman}
	
	\section{Revision History}
	
	\begin{tabularx}{\textwidth}{p{3cm}p{2cm}X} \toprule {\bf Date} & {\bf Version}
		& {\bf Notes}\\ \midrule Nov 23, 2020 & 1.0 & Initial Design\\ Dec 18, 2020 &
		1.1 & Revised Design\\ \bottomrule \end{tabularx}
	
	~\newpage
	
	\section{Symbols, Abbreviations and Acronyms}
	
	See SRS Documentation at
	\url{https://github.com/shmouses/SPDFM/blob/master/docs/SRS/SRS.pdf}.
	
	
	\newpage
	
	\tableofcontents
	
	\newpage
	
	\pagenumbering{arabic}
	
	\section{Introduction}
	
	The following document details the Module Interface Specifications for
	\progname{} program. \progname{} is a software for simulating surface plasmon
	enhanced electric field and current density in meshed geometry.
	
	Complementary documents include the System Requirement Specifications and
	Module Guide.  The full documentation and implementation can be found at
	\href{https://github.com/shmouses/SPDFM}{SPDFM repository} on github.
	
	\section{Notation}
	
	
	The structure of the MIS for modules comes from \citet{HoffmanAndStrooper1995},
	with the addition that template modules have been adapted from
	\cite{GhezziEtAl2003}.  The mathematical notation comes from Chapter 3 of
	\citet{HoffmanAndStrooper1995}.  For instance, the symbol := is used for a
	multiple assignment statement and conditional rules follow the form $(c_1
	\Rightarrow r_1 | c_2 \Rightarrow r_2 | ... | c_n \Rightarrow r_n )$.
	
	The following table summarizes the primitive data types used by \progname{}.
	
	\begin{center} \renewcommand{\arraystretch}{1.2} \noindent \begin{tabular}{l l
				p{7.5cm}} \toprule \textbf{Data Type} & \textbf{Notation} &
			\textbf{Description}\\ \midrule character & char & a single symbol or digit\\
			integer & $\mathbb{Z}$ & a number without a fractional component in (-$\infty$,
			$\infty$) \\ natural number & $\mathbb{N}$ & a number without a fractional
			component in [1, $\infty$) \\ real & $\mathbb{R}$ & any number in (-$\infty$,
			$\infty$)\\ imaginary& $\mathbb{I}$ & any number of form $i\times \mathbb{R}$
			where i is $\sqrt{-1}$ \\ \bottomrule \end{tabular} \end{center}
	
	\noindent The specification of \progname{} \ uses some derived data types:
	sequences, strings, and tuples. Sequences are lists filled with elements of the
	same data type. Strings are sequences of characters. Tuples contain a list of
	values, potentially of different types. In addition, \progname{} \ uses
	functions, which are defined by the data types of their inputs and outputs.
	Local functions are described by giving their type signature followed by their
	specification.
	
	\newpage \section{Module Decomposition}
	
	The following table is taken directly from the Module Guide document for this
	project.
	
	
	\begin{table}[h!] \centering \begin{tabular}{p{0.3\textwidth} p{0.7\textwidth}}
			\toprule \textbf{Level 1} & \textbf{Level 2}\\ \midrule
			
			{Hardware-Hiding Module} & ~ \\ \midrule
			
			\multirow{6}{0.3\textwidth}{Behaviour-Hiding Module} & \progname{} Control
			Module (M2, Section \ref{Module})\\ & Input Parameter Module (M5, Section
			\ref{IPM})\\ & Constant Parameters Module (M4, Section \ref{CPM})\\ & Mesh
			Input Module (M6, Section \ref{MIM})\\ & Frequency Domain FEM Solver Module
			(M7, Section \ref{FDSM})\\ & Output Module (M8, Section \ref{OM})\\ \midrule
			
			\multirow{1}{0.3\textwidth}{Software Decision Module} & Data Structure Module
			(M3, Section \ref{DSM})\\ \bottomrule
			
		\end{tabular} \caption{Module Hierarchy} \label{TblMH} \end{table}
	
	
	~\newpage
	
	\section{MIS of \progname{} Control Module} \label{Module}
	
	\subsection{Module} main
	
	\subsection{Uses} \begin{itemize} \item Data Structure (Section \ref{DSM})
		\item Input Parameter Module (Section \ref{IPM}) \item Mesh Input Module
		(Section \ref{MIM}) \item Frequency Domain FEM Solver Module (Section
		\ref{FDSM}) \item Output Module (Section \ref{OM})\\ \end{itemize}
	
	\subsection{Syntax}
	
	\subsubsection{Exported Constants} None. \subsubsection{Exported Access
		Programs}
	
	\begin{center} \begin{tabular}{p{2cm} p{4cm} p{4cm} p{2cm}} \hline
			\textbf{Name} & \textbf{In} & \textbf{Out} & \textbf{Exceptions} \\ \hline main
			& - & - & - \\ \hline \end{tabular} \end{center}
	
	\subsection{Semantics} \progname{} Control Module is designed to control the
	process flow in the software with respect to the SRS document. In this regard,
	current module organizes all other modules to satisfy the requirements that are
	specified in SRS. This module also help maintainability and expandability of
	\progname{} by classifying different parts of the code.
	
	\subsubsection{State Variables}
	
	None
	
	\subsubsection{Environment Variables}
	
	None
	
	\subsubsection{Assumptions}
	
	None
	
	\subsubsection{Access Routine Semantics}
	
	\subsubsection*{main():} \begin{itemize} \item transition: Control the order of
		execution of different modules:\\ \subitem Initiate the SPData object (M3,
		Section \ref{DSM}): doing so provides an empty framework that other modules can
		use to call or export data. \\ \subitem ParamLoad (M5, Section \ref{IPM}): This
		module inputs all the parameters and stores them in the Data object.\\
		
		\subitem MshInput (M6, Section \ref{MIM}): This module loads and prepares the
		mesh geometry for finite element calculations.\\
		
		\subitem FreqSolver (M7, Section \ref{FDSM}): Controls process flow and
		interactions with the FEniCS FEM solver. FEniCS is an external library used in
		this study for solving non-local hydrodynamic PDEs (see IM 2 in the SRS
		document). Interested readers can find more information about FEniCS at
		\cite{alnaes2015fenics,logg2012finite}.\\
		
		\subitem SPDoutput (M8, Section \ref{OM}): Exports the final results of the
		simulation into .vtk file.
		
		\item output: None \item exception: None \end{itemize}
	
	
	\subsubsection{Local Functions}
	
	None
	
	\newpage %
	%
	%
	%
	%
	\section{MIS of Input Parameter Module} \label{IPM}
	
	\subsection{Module} ParamLoad
	
	\subsection{Uses} \begin{itemize} \item Conatant Parameters Module (Section
		\ref{CPM}) \item Data Structure Module (Section \ref{DSM})
		
	\end{itemize}
	
	\subsection{Syntax}
	
	\subsubsection{Exported Constants} None
	
	\subsubsection{Exported Access Programs}
	
	\begin{center} \begin{tabular}{p{3cm} p{2cm} p{2cm} p{8cm}} \hline
			\textbf{Name} & \textbf{In} & \textbf{Out} & \textbf{Exceptions} \\ \hline
			ParamLoad & string & - & PolarizationRangeError, DirectionNormalityError,
			LightOrthogonalityError,  WavelengthRangeError, GammaRangeError\\
			
			\hline \end{tabular} \end{center}
	
	\subsection{Semantics}
	
	\subsubsection{State Variables}
	
	SPData object: stores input light source and material properties data
	
	\subsubsection{Environment Variables} ParamFile: A file containing sequence of
	strings that provides data related to the light source and material properties.
	
	\subsubsection{Assumptions} \begin{itemize}
		
		\item ParamLoad will be called before the values of any state variables will
		be accessed.
		
		\item The file contains the string equivalents of the numeric values for each
		input parameter in order, each on a new line. The order of the input data is
		as below (further explanation about input parameters can be found in ):
		
		\# Data in ParamLSfile:
		
		line 1: $p_x$, $p_y$, $p_z$ (light polarization)
		
		line 2: $d_x$, $d_y$, $d_z$ (direction of light propagation)
		
		line 3: $WL_{min}$ (the shortest wavelength of the light source)
		
		line 4: $WL_{max}$ (the longest wavelength of the light source)
		
		line 5: NumLS (number of wavelengths/light sources between min and max
		wavelength)
		
		line 6: eps0 (environment permittivity)
		
		line 7: mu0 (environment permeability)
		
		line 8: gamma (plasmon damping parameter)
		
		line 9: PlasmaFreq (plasma frequency)
		
		line 10: Beta (Fermi velocity proportionality)
		
		
	\end{itemize}
	
	\subsubsection{Access Routine Semantics}
	
	ParamLoad is a function to load, verify, and store input data (R1 and R2 from
	SRS).
	
	\subsubsection*{ParamLoad(pathLS, pathMP):}
	
	\begin{itemize} \item transition: pathLS (light source data) and pathMP
		(material properties) are the file paths for the input files which user
		provides. The following procedure is performed: \subitem -- Verify the format
		of the files to be .txt.\\ \subitem -- From ParamLSfile (located at pathLS)
		below parameters are obtained (As PDE equation (IM 2 SRS document) is being
		solve in the frequency domain light source should have a minimum and a maximum
		wavelength to specify the interval of the frequency domain. In this regard,
		number of frequencies in the intervals should be input as NumLS):
		
		\subsubitem Polarization of the incident light($\mathbb{R}^3$ vector):
		data.pol := p := [$p_x$, $p_y$, $p_z$] \subsubitem Direction of the incident
		light ($\mathbb{R}^3$ vector): data.dir := d := [$d_x$, $d_y$, $d_z$]
		\subsubitem Minimum wavelength of the light source ($\mathbb{R}$): data.WLmin
		:= WL$_{min}$ \subsubitem Maximum wavelength of the light source
		($\mathbb{R}$): data.WLmax := WL$_{max}$ \subsubitem Number of different
		wavelengths in the interval ($\mathbb{N}$): data.NumLS := NumLS \subsubitem
		Environment permittivity ($\mathbb{R}$): data.eps0 := eps0 \subsubitem
		Environment permeability ($\mathbb{R}$): data.mu0 := mu0 \subsubitem Plamson
		damping parameter ($\mathbb{R}$): data.damp := gamma \subsubitem Plasma
		frequency ($\mathbb{R}$): data.Pfreq := PlasmaFreq \subsubitem Fermi velocity
		proportionality ($\mathbb{R}$): data.beta := Beta\\ \subitem -- VerifyPol
		\subitem -- VerifyDir\\ \subitem -- VerifyWL\\ \subitem -- VerifyGamma\\
		
		\item output: None
		
		\item exception: exc :=\  \noindent \begin{longtable*}[l]{l l} \ \ \ \ \ \ If
			the file addressed by pathLS or path MP doesn't exist & $=>$ badFilePath\\ \ \
			\ \ \ \ If the file format is not .csv & $=>$ badFileFormat\\ \end{longtable*}
		
	\end{itemize}
	
	\subsubsection{Local Functions}
	
	\subsubsection*{verifyPol:} \begin{itemize} \item output: None \item exception:
		exc := \begin{longtable*}[l]{l l} \ \ \ \ \ \ $\neg(data.PminLmt< \|p\|<
			data.PmaxLmt)$ & $=>$ PolarizationRangeError\\ \end{longtable*}
		
	\end{itemize}
	
	\subsubsection*{verifyDir:} \begin{itemize} \item output: None \item exception:
		exc := \noindent \begin{longtable*}[l]{l l} \ \ \ \ \ \ $\|d\|\ \neq\ 1$ & $=>$
			DirectionRangeError\\ \ \ \ \ \ \ $d . p != 0$ & $=>$ LightOrthogonalityError\\
		\end{longtable*}
		
	\end{itemize}
	
	\subsubsection*{verifyWL:} \begin{itemize} \item output: None \item exception:
		exc := \noindent \begin{longtable*}[l]{l l} \ \ \ \ \ \ $\neg
			(data.WLminLmt<WL_{min}<WL_{max}<data.WLmaxLmt)$ &$=>$WavelengthRangeError\\
		\end{longtable*}
		
	\end{itemize}
	
	
	\subsubsection*{verifyGamma:} \begin{itemize} \item output: None \item
		exception: exc :=  \noindent \begin{longtable*}[l]{l l} \ \ \ \ \ \ $\neg
			(data.dampMinLmt <  gamma < data.dampMaxLmt)$ & $=>$ GammaRangeError\\
		\end{longtable*}
		
	\end{itemize}
	
	
	\newpage %
	%
	%
	%
	%
	\section{MIS of Constant Parameters Module} \label{CPM}
	
	
	\subsection{Module} Template Module
	
	\subsection{Uses} \begin{itemize} \item Hardware Hiding Module \end{itemize}
	
	\subsection{Syntax}
	
	\subsubsection{Exported Constants}
	
	None
	
	\subsubsection{Exported Access Programs}
	
	\begin{center} \begin{tabular}{p{2cm} p{4cm} p{4cm} p{2cm}} \hline
			\textbf{Name} & \textbf{In} & \textbf{Out} & \textbf{Exceptions} \\ \hline
			ConstParam & - & - & - \\ \hline \end{tabular} \end{center}
	
	\subsection{Semantics}
	
	\subsubsection{State Variables} ConstParam: An object containing real values.
	\begin{itemize} \item ConstParam.PminLmt $\in \mathbb{R}$ \item
		ConstParam.PmaxLmt $\in \mathbb{R}$ \item ConstParam.RadiusMinLmt $\in
		\mathbb{R}$ \item ConstParam.RadiusMaxLmt $\in \mathbb{R}$ \item
		ConstParam.WLminLmt $\in \mathbb{R}$ \item ConstParam.WLmaxLmt $\in \mathbb{R}$
		\item ConstParam.dampMinLmt $\in \mathbb{R}$ \item ConstParam.dampMaxLmt $\in
		\mathbb{R}$ \end{itemize}
	
	\subsubsection{Environment Variables}
	
	N/A \subsubsection{Assumptions}
	
	None
	
	\subsubsection{Access Routine Semantics} Store constant variables as indicated
	in Table 2 of the SRS (aligned with R2 of the SRS).
	
	\subsubsection*{ConstParam:} \begin{itemize} \item transition: Initiation of
		ConstParam object and stroing constant values obtained from Table 2 of the SRS
		in it: \subitem ConstParam.PminLmt := -100 \subitem ConstParam.PmaxLmt := 100
		\subitem ConstParam.RadiusMinLmt := 10 \subitem ConstParam.RadiusMaxLmt := 100
		\subitem ConstParam.WLminLmt := 200 \subitem ConstParam.WLmaxLmt := 1000
		\subitem ConstParam.dampMinLmt:= 0.01 \subitem ConstParam.dampMaxLmt:= 1
		
		\item output: None. \item exception: None. \end{itemize}
	
	
	\subsubsection{Local Functions}
	
	None
	
	\newpage %
	%
	%
	%
	
	\section{MIS of Mesh Input Module} \label{MIM}
	
	\subsection{Module} GmshInput
	
	\subsection{Uses} \begin{itemize} \item Data Structure (Section \ref{DSM})
		
	\end{itemize}
	
	\subsection{Syntax}
	
	\subsubsection{Exported Constants} None. \subsubsection{Exported Access
		Programs}
	
	\begin{center} \begin{tabular}{p{4cm} p{2cm} p{2cm} p{2cm}} \hline
			\textbf{Name} & \textbf{In} & \textbf{Out} & \textbf{Exceptions} \\ \hline
			GmshInput & string & - & FileError \\ \hline \end{tabular} \end{center}
	
	\subsection{Semantics}
	
	\subsubsection{State Variables}
	
	SPData object to store the mesh data.
	
	\subsubsection{Environment Variables} inputMesh: .xml files containing the data
	related to the meshed geometry (nodes, physical regions, boundaries).
	
	\subsubsection{Assumptions}
	
	None
	
	\subsubsection{Access Routine Semantics}
	
	This module satisfies R1 and R2 requirements from the SRS document.
	
	\subsubsection*{gmshInput(pathMESH):} \begin{itemize} \item transition:
		pathMESH is the file path for the input mesh file. The following procedure is
		performed: \subitem -- Load mesh data from the input file. \subitem  --
		Geometry is stored in the data structure: SPData.mesh := XMLmesh
		
		\item output: None \item exception: None \\
		
	\end{itemize}
	
	\subsubsection{Local Functions}
	
	None \newpage %
	%
	%
	%
	%
	\section{MIS of Frequency Domain FEM Solver Module:} \label{FDSM}
	
	
	\subsection{Module} FreqSolver
	
	
	\subsection{Uses} \begin{itemize} \item Data Structure Modules (Section
		\ref{DSM}) \end{itemize}
	
	\subsection{Syntax}
	
	\subsubsection{Exported Constants} None. \subsubsection{Exported Access
		Programs}
	
	\begin{center} \begin{tabular}{p{2cm} p{4cm} p{4cm} p{2cm}} \hline
			\textbf{Name} & \textbf{In} & \textbf{Out} & \textbf{Exceptions} \\ \hline
			FreqSolver & - & - & - \\ \hline \end{tabular} \end{center}
	
	\subsection{Semantics}
	
	\subsubsection{State Variables}
	
	SPData: the object containing all the data required for FEM solver
	
	\subsubsection{Environment Variables}
	
	None
	
	\subsubsection{Assumptions}
	
	None
	
	\subsubsection{Access Routine Semantics}
	
	This module specifically feeds the frequency domain PDE equations to the
	\hyperref{https://fenicsproject.org/}{}{}{FEniCS} finite element PDE solver.
	FEniCS toolbox is used for all finite element setups in \progname{}. To better
	understand the  FEniCS implementation and data structure, readers are
	encouraged to look up \cite{alnaes2015fenics} and \cite{logg2012finite}. The
	FEniCS-related implementations which include defining function space, element
	space, trial function, test function, and inputting variational form of the
	system of equations are not discussed in this document as these descriptions
	are part of FEniCS toolbox and are discussed in details in references mentioned
	above.
	
	\subsubsection*{FreqSolver():} \begin{itemize} \item transition: FEniCS toolbox
		will receive nonlocal hydrodynamic equation system (IM2 SRS document) to
		calculate the electric field and electric current at different given
		frequencies and the mesh.
		
		\subitem -- Setting up FEniCS environment.
		
		\subitem -- importing below equations to the FEniCS, all the parameters in the
		equation are defined in IM2 and also the table of symbols in the SRS document:
		\begin{equation*} \begin{gathered} -\int_\Omega
				\beta^2(\nabla.\psi)(\nabla.\textbf{J}_{HD})dV+\omega(\omega+i\gamma)\int_{\Omega} \psi. \textbf{J}_HD dV - \ i\omega \omega^2_p \int_\Omega \psi.\varepsilon_{0}\textbf{E}dV = 0 \\ \\ \int_\Omega ((\nabla \times \phi) . (\mu^{-1}_0 \nabla \times \textbf{E})-\omega^2 \phi.\epsilon_{local} \textbf{E}) dV + \ \int_{\partial \Omega} \phi.(\textbf{n} \times (\mu^{-1}_0 \nabla \times \textbf{E}))dA = i\omega \int_\Omega \phi. \textbf{J}_{HD} dV \\ \\Boundary\ Conditions: \\ \int_\Omega ((\nabla \times \phi).(\mu^{-1}_{0} \nabla \times \textbf{E})- \omega^2\phi \epsilon_{local} \textbf{E}_i)dV + \int_{\partial \Omega} \phi . DtN(\textbf{E})dA \ \ - i\omega \int_\Omega \phi . \textbf{J}_{HD}dV = \\ -\int_{\partial \Omega} \phi.(n \times (\mu^{-1}_0 \nabla \times \textbf{E}_i))dA + \int_{\partial \Omega} \phi.DtN(\textbf{E}_i)dA \\ \\ n.\textbf{J}_{HD}=0 \ \ on \ \partial \Omega \end{gathered} \end{equation*} \\ \subitem \# For importing the data to FEniCS, as complex numbers are not defined in this toolbox parameters and equations should be separated into real and imaginary parts. \begin{equation*} \begin{gathered} \textbf{J}_{HD}=\textbf{J}^{real}_{HD} + i\textbf{J}^{img}_{HD} \\ \textbf{E} = \textbf{E}^{real} + i\textbf{E}^{img} \end{gathered} \end{equation*}
		
		
		\subitem --Store the result for each frequency in the SPData.FEMresult.
		
		\begin{equation*} SPData.FEMresult = \begin{pmatrix} \textbf{E}^{real}\\
				\textbf{E}^{img}\\ \textbf{J}^{real}_{HD}\\ \textbf{J}^{img}_{HD}
		\end{pmatrix} \end{equation*}
		
	\end{itemize}
	
	\newpage %
	%
	%
	%
	\section{MIS of Data Structure Module} \label{DSM}
	
	\subsection{Module} Template Module
	
	\subsection{Uses} \begin{itemize} \item Hardware Hiding Module \end{itemize}
	
	\subsection{Syntax}
	
	\subsubsection{Exported Constants} None. \subsubsection{Exported Access
		Programs}
	
	\begin{center} \begin{tabular}{p{2cm} p{4cm} p{4cm} p{2cm}} \hline
			\textbf{Name} & \textbf{In} & \textbf{Out} & \textbf{Exceptions} \\ \hline init
			& - & - & - \\ \hline \end{tabular} \end{center}
	
	\subsection{Semantics}
	
	\subsubsection{State Variables}
	
	data:object \begin{itemize} \item data.pol $\in$ $\mathbb{R}^3$ \item data.dir
		$\in$ $\mathbb{R}^3$ \item data.PminLmt $\in$ $\mathbb{R}$ \item data.PmaxLmt
		$\in$ $\mathbb{R}$ \item data.WLmin $\in \mathbb{R}$ \item data.WLmax $\in
		\mathbb{R}$ \item data.WLminLmt $\in \mathbb{R}$ \item data.WLmaxLmt $\in
		\mathbb{R}$ \item data.NumLS $\in \mathbb{N}$ \item data.eps0 $\in \mathbb{R}$
		\item data.mu0  $\in \mathbb{R}$ \item data.beta $\in \mathbb{R}$ \item
		data.damp $\in \mathbb{R}$ \item data.dampMinLmt $\in \mathbb{R}$ \item
		data.dampMaxLmt $\in \mathbb{R}$ \item data.Pfreq $\in \mathbb{R}$ \item
		data.XDFMmesh : Mesh object \item data.FreqRes $\in \mathbb{R}^4$
		
		
	\end{itemize}
	
	\subsubsection{Environment Variables}
	
	N/A \subsubsection{Assumptions}
	
	None
	
	\subsubsection{Access Routine Semantics} N/A
	
	\subsubsection{Local Functions}
	
	None
	
	\newpage %
	%
	%
	%
	\section{MIS of Output Module} \label{OM}
	
	\subsection{Module} Output
	
	\subsection{Uses} \begin{itemize} \item Hardware Hiding Module \item Data
		Structure Module (Section \ref{DSM})
		
	\end{itemize}
	
	\subsection{Syntax}
	
	\subsubsection{Exported Constants} None. \subsubsection{Exported Access
		Programs}
	
	\begin{center} \begin{tabular}{p{2cm} p{4cm} p{4cm} p{2cm}} \hline
			\textbf{Name} & \textbf{In} & \textbf{Out} & \textbf{Exceptions} \\ \hline
			Output& - & string (.pvd and .vtk file) & - \\
			
			\hline \end{tabular} \end{center}
	
	\subsection{Semantics}
	
	\subsubsection{State Variables} None
	
	\subsubsection{Environment Variables} None
	
	\subsubsection{Assumptions} None
	
	\subsubsection{Access Routine Semantics}
	
	\subsubsection*{Output():} \begin{itemize} \item transition: None \item output:
		export .pvd and .vtk files from SPData.FEMresult \item exception: None
	\end{itemize}
	
	
	\subsubsection{Local Functions} None
	
	\newpage %
	%
	%
	%
	%
	%
	%
	%
	%
	%
	
	
	\bibliographystyle {plainnat} \bibliography {../../../refs/References}
	
	%\newpage
	
	%\section{Appendix} \label{Appendix}
	
	%$\wss{Extra information if required}
\end{document}