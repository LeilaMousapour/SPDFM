\documentclass[12pt, titlepage]{article}

\usepackage{amsmath, mathtools}

\usepackage[round]{natbib} \usepackage{amsfonts} \usepackage{amssymb}
\usepackage{graphicx} \usepackage{colortbl} \usepackage{xr}
\usepackage{hyperref} \usepackage{longtable} \usepackage{xfrac}
\usepackage{tabularx} \usepackage{float} \usepackage{siunitx}
\usepackage{booktabs} \usepackage{multirow} \usepackage[section]{placeins}
\usepackage{caption} \usepackage{fullpage}

\hypersetup{ bookmarks=true,     % show bookmarks bar?
	colorlinks=true,       % false: boxed links; true: colored links
	linkcolor=red,          % color of internal links (change box color with
	% linkbordercolor)
	citecolor=blue,      % color of links to bibliography
	filecolor=magenta,  % color of file links
	urlcolor=cyan          % color of external links
}

\usepackage{array}

\externaldocument{../../SRS/SRS}

%% Comments

\usepackage{color}

\newif\ifcomments\commentstrue

\ifcomments
\newcommand{\authornote}[3]{\textcolor{#1}{[#3 ---#2]}}
\newcommand{\todo}[1]{\textcolor{red}{[TODO: #1]}}
\else
\newcommand{\authornote}[3]{}
\newcommand{\todo}[1]{}
\fi

\newcommand{\wss}[1]{\authornote{blue}{SS}{#1}} 
\newcommand{\plt}[1]{\authornote{magenta}{TPLT}{#1}} %For explanation of the template
\newcommand{\an}[1]{\authornote{cyan}{Author}{#1}}
 %% Common Parts

\newcommand{\progname}{SPDFM} % PUT YOUR PROGRAM NAME HERE %Every program
                                % should have a name


\begin{document}
	
	\title{Module Interface Specification for \progname{}}
	
	\author{S. Shayan Mousavi M.}
	
	\date{\today}
	
	\maketitle
	
	\pagenumbering{roman}
	
	\section{Revision History}
	
	\begin{tabularx}{\textwidth}{p{3cm}p{2cm}X} \toprule {\bf Date} & {\bf Version}
		& {\bf Notes}\\ \midrule Nov 23, 2020 & 1.0 & Notes\\ \bottomrule \end{tabularx}
	
	~\newpage
	
	\section{Symbols, Abbreviations and Acronyms}
	
	See SRS Documentation at
	\url{https://github.com/shmouses/SPDFM/blob/master/docs/SRS/SRS.pdf}.
	
	
	\newpage
	
	\tableofcontents
	
	\newpage
	
	\pagenumbering{arabic}
	
	\section{Introduction}
	
	The following document details the Module Interface Specifications for
	\progname{} program. \progname{} is a software for simulating surface plasmon
	enhanced electric field and current density in meshed geometry.
	
	Complementary documents include the System Requirement Specifications and Module
	Guide.  The full documentation and implementation can be found at
	\href{https://github.com/shmouses/SPDFM}{SPDFM repository} on github.
	
	\section{Notation}
	
	
	The structure of the MIS for modules comes from \citet{HoffmanAndStrooper1995},
	with the addition that template modules have been adapted from
	\cite{GhezziEtAl2003}.  The mathematical notation comes from Chapter 3 of
	\citet{HoffmanAndStrooper1995}.  For instance, the symbol := is used for a
	multiple assignment statement and conditional rules follow the form $(c_1
	\Rightarrow r_1 | c_2 \Rightarrow r_2 | ... | c_n \Rightarrow r_n )$.
	
	The following table summarizes the primitive data types used by \progname{}.
	
	\begin{center} \renewcommand{\arraystretch}{1.2} \noindent \begin{tabular}{l l
				p{7.5cm}} \toprule \textbf{Data Type} & \textbf{Notation} &
			\textbf{Description}\\ \midrule character & char & a single symbol or digit\\
			integer & $\mathbb{Z}$ & a number without a fractional component in (-$\infty$,
			$\infty$) \\ natural number & $\mathbb{N}$ & a number without a fractional
			component in [1, $\infty$) \\ real & $\mathbb{R}$ & any number in (-$\infty$,
			$\infty$)\\ imaginary& $\mathbb{I}$ & any number of form $i\times \mathbb{R}$
			where i is $\sqrt{-1}$ \\ \bottomrule \end{tabular} \end{center}
	
	\noindent The specification of \progname{} \ uses some derived data types:
	sequences, strings, and tuples. Sequences are lists filled with elements of the
	same data type. Strings are sequences of characters. Tuples contain a list of
	values, potentially of different types. In addition, \progname{} \ uses
	functions, which are defined by the data types of their inputs and outputs.
	Local functions are described by giving their type signature followed by their
	specification.
	
	\newpage \section{Module Decomposition}
	
	The following table is taken directly from the Module Guide document for this
	project.
	
	
	\begin{table}[h!] \centering \begin{tabular}{p{0.3\textwidth} p{0.6\textwidth}}
			\toprule \textbf{Level 1} & \textbf{Level 2}\\ \midrule
			
			{Hardware-Hiding Module} & ~ \\ \midrule
			
			\multirow{7}{0.3\textwidth}{Behaviour-Hiding Module} & \progname{} Control
			Module\\ & Input Parameter Module\\ & Constant Parameters Module\\ & Mesh
			Input Module\\ & SPD Simulator Module\\ & Output Module\\ \midrule
			
			\multirow{3}{0.3\textwidth}{Software Decision Module} & Frequency Domain PDE
			Solver Module\\ & Data Structure Module\\ \bottomrule
			
		\end{tabular} \caption{Module Hierarchy} \label{TblMH} \end{table}
	
	
	~\newpage
	
	\section{MIS of \progname{} Control Module} \label{Module}
	
	\subsection{Module} main
	
	\subsection{Uses} \begin{itemize} \item Data Structure (Section \ref{DSM}) \item
		Input Parameter Module (Section \ref{IPM}) \item Mesh Input Module (Section
		\ref{MIM}) \item SPD Simulator Control Module (Section \ref{SSM}) \item Output
		Module (Section \ref{OM})\\ \end{itemize}
	
	\subsection{Syntax}
	
	\subsubsection{Exported Constants} None. \subsubsection{Exported Access
		Programs}
	
	\begin{center} \begin{tabular}{p{2cm} p{4cm} p{4cm} p{2cm}} \hline \textbf{Name}
			& \textbf{In} & \textbf{Out} & \textbf{Exceptions} \\ \hline main & - & - & - \\
			\hline \end{tabular} \end{center}
	
	\subsection{Semantics} \progname{} Control Module is design to control the
	process flow in the software. With respect to the SRS document, current module
	organizes all other modules to satisfy all the requirements that specified in
	SRS. This module also help maintainability and expandability of \progname{} by
	classifying different parts of the code.
	
	\subsubsection{State Variables}
	
	None
	
	\subsubsection{Environment Variables}
	
	None
	
	\subsubsection{Assumptions}
	
	None
	
	\subsubsection{Access Routine Semantics}
	
	\subsubsection*{main():} \begin{itemize} \item transition: Control the flow
		input data, calculation, and the output data by following below steps:\\
		
		\subitem Initiate global data object: doing so provides an empty framework that
		lets user store data at different modules. \\ \subitem ParamLoad: This module
		inputs all the parameters and stores them in the data structure.\\
		
		\subitem MshInput: This module loads and prepares the mesh geometry for finite
		element calculations.\\
		
		\subitem SPDsimulator: Controls process flow and interactions with the FEniCS
		PDE solver. FEniCS PDE solver is a library used in this study for solving
		non-local hydrodynamic PDEs (see IM 2 in the SRS document). Interested readers
		can find more information about FEniCS at
		\cite{alnaes2015fenics,logg2012finite}.\\
		
		\subitem SPDoutput: Exports the final results of the simulation into .vtk file.
		
		\item output: None \item exception: None \end{itemize}
	
	
	\subsubsection{Local Functions}
	
	None
	
	\newpage %
	%
	%
	%
	%
	\section{MIS of Input Parameter Module} \label{IPM}
	
	\subsection{Module} ParamLoad
	
	\subsection{Uses} \begin{itemize} \item Specification Parameters Module (Section
		\ref{CPM}) \item Data Structure (Section \ref{DSM})
		
	\end{itemize}
	
	\subsection{Syntax}
	
	\subsubsection{Exported Constants} None
	
	\subsubsection{Exported Access Programs}
	
	\begin{center} \begin{tabular}{p{3cm} p{2cm} p{2cm} p{8cm}} \hline \textbf{Name}
			& \textbf{In} & \textbf{Out} & \textbf{Exceptions} \\ \hline ParamLoad & string
			& - & FileError\\ VrifyParam & - & - & PolarizationRangeError,
			DirectionNormalityError, LightOrthogonalityError,  WavelengthRangeError,
			GammaRangeError\\
			
			\hline \end{tabular} \end{center}
	
	\subsection{Semantics}
	
	\subsubsection{State Variables}
	
	data: object
	
	\subsubsection{Environment Variables} ParamLSfile: A file containing sequence of
	strings that provides data related to the light source. \\ ParamMPfile: A file
	containing sequence of strings that provides data related to the materials
	properties.
	
	\subsubsection{Assumptions} \begin{itemize}
		
		\item ParamLoad will be called before the values of any state variables will be
		accessed.
		
		\item The file contains the string equivalents of the numeric values for each
		input parameter in order, each on a new line. The order of the input data is as
		below:
		
		\# Data in ParamLSfile:
		
		line 1: $p_x$, $p_y$, $p_z$
		
		line 2: $d_x$, $d_y$, $d_z$
		
		line 3: $WL_{min}$
		
		line 4: $WL_{max}$
		
		line 5: NumLS
		
		\# Data in ParamMPile:
		
		line 1: eps0
		
		line 2: mu0
		
		line 3: gamma
		
		line 4: PlasmaFreq
		
		line 5: Beta
		
		
	\end{itemize}
	
	\subsubsection{Access Routine Semantics}
	
	Function to load, verify, and store input data (R1 and R2 from SRS).
	
	\subsubsection*{ParamLoad(pathLS, pathMP):}
	
	\begin{itemize} \item transition: pathLS (light source data) and pathMP
		(material properties) are the file paths for the input files which user
		provides. The following procedure is performed: \subitem -- Verify the format of
		the files to be .csv.\\ \subitem -- From ParamLSfile (located at pathLS) below
		parameters are obtained (As PDE equation (IM 2 SRS document) is being solve in
		the frequency domain light source should have a minimum and a maximum wavelength
		to specify the interval of the frequency domain. In this regard, number of
		frequencies in the intervals should be input as NumLS):
		
		\subsubitem Polarization of the incident light($\mathbb{R}^3$ vector): data.pol
		:= p := [$p_x$, $p_y$, $p_z$]
		
		\subsubitem Direction of the incident light ($\mathbb{R}^3$ vector): data.dir
		:= d := [$d_x$, $d_y$, $d_z$]
		
		
		\subsubitem Minimum wavelength of the light source ($\mathbb{R}$): data.WLmin
		:= WL$_{min}$
		
		\subsubitem Maximum wavelength of the light source ($\mathbb{R}$): data.WLmax
		:= WL$_{max}$
		
		\subsubitem Number of different wavelengths in the interval ($\mathbb{N}$):
		data.NumLS := NumLS \\
		
		\subitem -- verifyPol\\ \subitem -- verifyDir\\ \subitem -- verifyWL\\
		
		
		\subitem --From ParamMSfile below information is obtained:
		
		\subsubitem Environment permittivity ($\mathbb{R}$): data.eps0 := eps0
		
		\subsubitem Environment permeability ($\mathbb{R}$): data.mu0 := mu0
		\subsubitem Plamson damping parameter ($\mathbb{R}$): data.damp := gamma
		\subsubitem Plasma frequency ($\mathbb{R}$): data.Pfreq := PlasmaFreq
		\subsubitem Fermi velocity proportionality ($\mathbb{R}$): data.beta := Beta\\
		
		\subitem -- verifyGamma\\ \item output: None
		
		\item exception: exc :=\  \noindent \begin{longtable*}[l]{l l} \ \ \ \ \ \ If the file
			addressed by pathLS or path MP doesn't exist & $=>$ badFilePath\\ \ \ \ \ \ \
			If the file format is not .csv & $=>$ badFileFormat\\ \end{longtable*}
		
	\end{itemize}
	
	\subsubsection{Local Functions}
	
	\subsubsection*{verifyPol:} \begin{itemize} \item output: None \item exception: exc := 
		\begin{longtable*}[l]{l l} \ \ \ \ \ \ $\neg(data.PminLmt< \|p\|< data.PmaxLmt)$
			& $=>$ PolarizationRangeError\\ \end{longtable*}
		
	\end{itemize}
	
	\subsubsection*{verifyDir:} \begin{itemize} \item output: None \item exception: exc := 
		\noindent \begin{longtable*}[l]{l l} \ \ \ \ \ \ $\|d\|\ \neq\ 1$ & $=>$
			DirectionRangeError\\ \ \ \ \ \ \ $d . p != 0$ & $=>$ LightOrthogonalityError\\
		\end{longtable*}
		
	\end{itemize}
	
	\subsubsection*{verifyWL:} \begin{itemize} \item output: None \item exception: exc := 
		\noindent \begin{longtable*}[l]{l l} \ \ \ \ \ \ $\neg
			(data.WLminLmt<WL_{min}<WL_{max}<data.WLmaxLmt)$ &$=>$WavelengthRangeError\\
		\end{longtable*}
		
	\end{itemize}
	
	
	\subsubsection*{verifyGamma:} \begin{itemize} \item output: None \item
		exception: exc :=  \noindent \begin{longtable*}[l]{l l} \ \ \ \ \ \ $\neg
			(data.dampMinLmt <  gamma < data.dampMaxLmt)$ & $=>$ GammaRangeError\\
		\end{longtable*}
		
	\end{itemize}
	
	
	\newpage %
	%
	%
	%
	%
	\section{MIS of Constant Parameters Module} \label{CPM}
	
	
	\subsection{Module} ConstParam
	
	\subsection{Uses} \begin{itemize} \item Data Structure (Section \ref{DSM})
	\end{itemize}
	
	\subsection{Syntax}
	
	\subsubsection{Exported Constants}
	
	From Table 2 in SRS update the values in the data object:\\
	
	data.PminLmt := -10\\
	
	data.PmaxLmt := 10\\
	
	data.RadiusMinLmt := 10\\
	
	data.RadiusMaxLmt := 100\\
	
	data.WLminLmt := 200\\
	
	data.WLmaxLmt := 1000\\
	
	data.dampMinLmt:= 0.01\\
	
	data.dampMaxLmt:= 1\\
	
	Although currently only constants in the \progname{} are exported from
	specification parameters table in SRS document (Table 2), this module also
	considers future expansions of the program. In this regard, more constants can
	be added to this module in the future if needed.
	
	\subsubsection{Exported Access Programs}
	
	\begin{center} \begin{tabular}{p{2cm} p{4cm} p{4cm} p{2cm}} \hline \textbf{Name}
			& \textbf{In} & \textbf{Out} & \textbf{Exceptions} \\ \hline ConstParam & - &
-
			& - \\ \hline \end{tabular} \end{center}
	
	\subsection{Semantics} N/A
	
	\newpage %
	%
	%
	%
	
	\section{MIS of Mesh Input Module} \label{MIM}
	
	\subsection{Module} GmshInput
	
	\subsection{Uses} \begin{itemize} \item Data Structure (Section \ref{DSM})
		
	\end{itemize}
	
	\subsection{Syntax}
	
	\subsubsection{Exported Constants} None. \subsubsection{Exported Access
		Programs}
	
	\begin{center} \begin{tabular}{p{4cm} p{2cm} p{2cm} p{2cm}} \hline \textbf{Name}
			& \textbf{In} & \textbf{Out} & \textbf{Exceptions} \\ \hline GmshInput & string
			& - & FileError \\ MeshConvert & object & object & -\\ \hline \end{tabular}
	\end{center}
	
	\subsection{Semantics}
	
	\subsubsection{State Variables}
	
	For inputting the mesh object in this module,
	\hyperref{https://pypi.org/project/meshio/}{library}{}{meshio} toolbox is used.
	Interested readers for better understanding properties of meshio mesh object are
	referred to \cite{schlomer2019meshio} and \cite{dai2017high}. \\
	
	Data: mesh object
	
	\subsubsection{Environment Variables} inputMesh: A .msh file containing the data
	related to the meshed geometry.
	
	\subsubsection{Assumptions}
	
	None
	
	\subsubsection{Access Routine Semantics}
	
	Function to load, convert and verify the mesh file. This module satisfies R1 and
	R2 requirements from the SRS document.
	
	\subsubsection*{gmshInput(pathMESH):} \begin{itemize} \item transition: pathMESH
		is the file path for the input mesh file. The following procedure is performed:
		\subitem -- Verify the format of the file to be .msh. \subitem -- Load mesh
		object, GMESH, from the input file. \subitem -- MeshConvert \subitem -- Geometry
		is stored in the data structure: data.XDMFmesh := XDMFmesh
		
		\item output: None \item exception: exc :=  \noindent \begin{longtable*}[l]{l l} \ \ \
			\ \ \ If the file addressed by pathMESH doesn't exist & $=>$ badMeshFilePath\\
			\ \ \ \ \ \ If the file format is not .msh & $=>$ badMeshFileFormat\\
		\end{longtable*}
		
		
	\end{itemize}
	
	
	\subsubsection*{MeshConvert(GmeshFile):} \begin{itemize} \item transition: None
		\item output: XDMFmesh which is converted GmeshFile (.mesh format) into .xdmf
		format (which is an acceptable format for FEniCS to initiate FEM solver) \item
		exception: None \end{itemize}
	
	
	\subsubsection{Local Functions}
	
	None \newpage %
	%
	%
	%
	\section{MIS of SPD Simulator Module} \label{SSM}
	
	
	\subsection{Module} SPDSimulator
	
	\subsection{Uses} \begin{itemize} \item Frequency Domain PDE Solver Module
		(Section \ref{FDSM}) \end{itemize}
	
	\subsection{Syntax}
	
	\subsubsection{Exported Constants} None. \subsubsection{Exported Access
		Programs}
	
	\begin{center} \begin{tabular}{p{4cm} p{4cm} p{4cm} p{2cm}} \hline \textbf{Name}
			& \textbf{In} & \textbf{Out} & \textbf{Exceptions} \\ \hline SPDSimulator & - &
			- & - \\ \hline \end{tabular} \end{center}
	
	\subsection{Semantics}
	
	\subsubsection{State Variables}
	
	Data: object
	
	\subsubsection{Environment Variables} N/A
	
	\subsubsection{Assumptions} None.
	
	\subsubsection{Access Routine Semantics} Functions to calculate the Electric
	field and Electric Current density in the given mesh and illumination (satisfies
	R3, R4 from SRS document). In fact, this module will control flow and order of
	the different PDE solver units. Although at the moment it seems like this module
	only calls "Frequency Domain PDE Solver Module (Section \ref{FDSM})", it gives
	expandability to the software to have more PDE solver modules in the future.
	This module is aligned with the second non-functional requirement (NR2) from the
	SRS document which is maintainability and expandability of the software.
	
	\subsubsection*{SPDSimulator():} \begin{itemize} \item transition: \subitem --
		FreqSolver() \item output: None. \item exception: None. \end{itemize}
	
	\subsubsection{Local Functions} None
	
	
	
	\newpage %
	%
	%
	%
	\section{MIS of Frequency Domain PDE Solver Module:} \label{FDSM}
	
	
	\subsection{Module} FreqSolver
	
	
	\subsection{Uses} \begin{itemize} \item Data Structure Modules (Section
		\ref{DSM}) \end{itemize}
	
	\subsection{Syntax}
	
	\subsubsection{Exported Constants} None. \subsubsection{Exported Access
		Programs}
	
	\begin{center} \begin{tabular}{p{2cm} p{4cm} p{4cm} p{2cm}} \hline \textbf{Name}
			& \textbf{In} & \textbf{Out} & \textbf{Exceptions} \\ \hline FreqSolver & - & -
			& - \\ \hline \end{tabular} \end{center}
	
	\subsection{Semantics}
	
	\subsubsection{State Variables}
	
	data: object
	
	\subsubsection{Environment Variables}
	
	None
	
	\subsubsection{Assumptions}
	
	None
	
	\subsubsection{Access Routine Semantics}
	
	This module specifically feeds the frequency domain PDE equations to the
	\hyperref{https://fenicsproject.org/}{}{}{FEniCS} finite element PDE solver.
	FEniCS toolbox is used for all finite element setups in \progname{}. To better
	understand the  FEniCS implementation and data structure, readers are encouraged
	to look up \cite{alnaes2015fenics} and \cite{logg2012finite}. The FEniCS-related
	implementations which include defining function space, element space, trial
	function, test function, and inputting variational form of the system of
	equations are not discussed in this document as these descriptions are part of
	FEniCS toolbox and are discussed in details in references mentioned above.
	
	\subsubsection*{FreqSolver():} \begin{itemize} \item transition: FEniCS toolbox
		will receive nonlocal hydrodynamic equation system (IM2 SRS document) to
		calculate the electric field and electric current at different given frequencies
		and the mesh.
		
		\subitem -- Setting up FEniCS environment.
		
		\subitem -- importing below equations to the FEniCS, all the parameters in the
		equation are defined in IM2 and also the table of symbols in the SRS document:
		\begin{equation*} \begin{gathered} -\int_\Omega
				\beta^2(\nabla.\psi)(\nabla.\textbf{J}_{HD})dV+\omega(\omega+i\gamma)\int_{\Omega} \psi. \textbf{J}_HD dV - \ i\omega \omega^2_p \int_\Omega \psi.\varepsilon_{0}\textbf{E}dV = 0 \\ \\ \int_\Omega ((\nabla \times \phi) . (\mu^{-1}_0 \nabla \times \textbf{E})-\omega^2 \phi.\epsilon_{local} \textbf{E}) dV + \ \int_{\partial \Omega} \phi.(\textbf{n} \times (\mu^{-1}_0 \nabla \times \textbf{E}))dA = i\omega \int_\Omega \phi. \textbf{J}_{HD} dV \\ \\Boundary\ Conditions: \\ \int_\Omega ((\nabla \times \phi).(\mu^{-1}_{0} \nabla \times \textbf{E})- \omega^2\phi \epsilon_{local} \textbf{E}_i)dV + \int_{\partial \Omega} \phi . DtN(\textbf{E})dA \ \ - i\omega \int_\Omega \phi . \textbf{J}_{HD}dV = \\ -\int_{\partial \Omega} \phi.(n \times (\mu^{-1}_0 \nabla \times \textbf{E}_i))dA + \int_{\partial \Omega} \phi.DtN(\textbf{E}_i)dA \\ \\ n.\textbf{J}_{HD}=0 \ \ on \ \partial \Omega \end{gathered} \end{equation*} \\ \subitem \# For importing the data to FEniCS, as complex numbers are not defined in this toolbox parameters and equations should be separated into real and imaginary parts. \begin{equation*} \begin{gathered} \textbf{J}_{HD}=\textbf{J}^{real}_{HD} + i\textbf{J}^{img}_{HD} \\ \textbf{E} = \textbf{E}^{real} + i\textbf{E}^{img} \end{gathered} \end{equation*}
		
		
		\subitem --Store the result for each frequency in the Data.FDres.
		
		\begin{equation*} data.FDres = \begin{pmatrix} \textbf{E}^{real}\\
				\textbf{E}^{img}\\ \textbf{J}^{real}_{HD}\\ \textbf{J}^{img}_{HD} \end{pmatrix}
		\end{equation*}
		
	\end{itemize}
	
	\newpage %
	%
	%
	%
	\section{MIS of Data Structure Module} \label{DSM}
	
	\subsection{Module} data
	
	\subsection{Uses} \begin{itemize} \item Hardware Hiding module \end{itemize}
	
	\subsection{Syntax}
	
	\subsubsection{Exported Constants} None. \subsubsection{Exported Access
		Programs}
	
	\begin{center} \begin{tabular}{p{2cm} p{4cm} p{4cm} p{2cm}} \hline \textbf{Name}
			& \textbf{In} & \textbf{Out} & \textbf{Exceptions} \\ \hline init & - & - & - \\
			\hline \end{tabular} \end{center}
	
	\subsection{Semantics}
	
	\subsubsection{State Variables}
	
	data:object \begin{itemize} \item data.pol $\in$ $\mathbb{R}^3$ \item data.dir
		$\in$ $\mathbb{R}^3$ \item data.PminLmt $\in$ $\mathbb{R}$ \item data.PmaxLmt
		$\in$ $\mathbb{R}$ \item data.WLmin $\in \mathbb{R}$ \item data.WLmax $\in
		\mathbb{R}$ \item data.WLminLmt $\in \mathbb{R}$ \item data.WLmaxLmt $\in
		\mathbb{R}$ \item data.NumLS $\in \mathbb{N}$ \item data.eps0 $\in \mathbb{R}$
		\item data.mu0  $\in \mathbb{R}$ \item data.beta $\in \mathbb{R}$ \item
		data.damp $\in \mathbb{R}$ \item data.dampMinLmt $\in \mathbb{R}$ \item
		data.dampMaxLmt $\in \mathbb{R}$ \item data.Pfreq $\in \mathbb{R}$ \item
		data.XDFMmesh : Mesh object \item data.FreqRes $\in \mathbb{R}^4$
		
		
	\end{itemize}
	
	\subsubsection{Environment Variables}
	
	N/A \subsubsection{Assumptions}
	
	None
	
	\subsubsection{Access Routine Semantics} N/A
	
	\subsubsection{Local Functions}
	
	None
	
	\newpage %
	%
	%
	%
	\section{MIS of Output Module} \label{OM}
	
	\subsection{Module} Output
	
	\subsection{Uses} \begin{itemize} \item Hardware Hiding Module \item Data
		Structure Module (Section \ref{DSM})
		
	\end{itemize}
	
	\subsection{Syntax}
	
	\subsubsection{Exported Constants} None. \subsubsection{Exported Access
		Programs}
	
	\begin{center} \begin{tabular}{p{2cm} p{4cm} p{4cm} p{2cm}} \hline \textbf{Name}
			& \textbf{In} & \textbf{Out} & \textbf{Exceptions} \\ \hline Output& - & - & -
			\\ vtkExport & - & string (.vtk file) & - \\
			
			\hline \end{tabular} \end{center}
	
	\subsection{Semantics}
	
	\subsubsection{State Variables} None
	
	\subsubsection{Environment Variables} None
	
	\subsubsection{Assumptions} None
	
	\subsubsection{Access Routine Semantics}
	
	\subsubsection*{Output():} \begin{itemize} \item transition: Control different
		output units. The current version of the code only uses one unit as below:
		\subitem -- vtksaver \item output: None \item exception: None \end{itemize}
	
	\subsubsection*{vtkSaver():} \begin{itemize} \item transition: None \item
		output: Exported vtkFile: \subitem vtkFile := vtk export of data.FreqRes \item
		exception: None \end{itemize}
	
	\subsubsection{Local Functions} None
	
	\newpage %
	%
	%
	%
	%
	%
	%
	%
	%
	%
	
	
	\bibliographystyle {plainnat} \bibliography {../../../refs/References}
	
	%\newpage
	
	%\section{Appendix} \label{Appendix}
	
	%$\wss{Extra information if required}
\end{document}