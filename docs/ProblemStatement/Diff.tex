%DIF 1-3c1
%DIF LATEXDIFF DIFFERENCE FILE
%DIF DEL ProblemStatement_old.tex   Sat Sep 26 01:04:37 2020
%DIF ADD ProblemStatement.tex       Sat Sep 26 02:05:41 2020
%DIF < \documentclass{article}
%DIF < \usepackage{tabularx}
%DIF < \usepackage{booktabs}{\tiny }
%DIF -------
\documentclass{article} \usepackage{tabularx} \usepackage{booktabs} %DIF >
%DIF -------

\title{CAS 741: Problem Statement\\Surface Plasmon Dynamics Finite Method}

\author{S. Shayan Mousavi M.\\github: @shmouses}

\date{2020/09/\DIFdelbegin \DIFdel{20}\DIFdelend \DIFaddbegin
	\DIFadd{24}\DIFaddend }

\input{Comments} %DIF PREAMBLE EXTENSION ADDED BY LATEXDIFF
%DIF UNDERLINE PREAMBLE %DIF PREAMBLE
\RequirePackage[normalem]{ulem} %DIF PREAMBLE
\RequirePackage{color}\definecolor{RED}{rgb}{1,0,0}\definecolor{BLUE}{rgb}{0,0,1} %DIF PREAMBLE
\providecommand{\DIFadd}[1]{{\protect\color{blue}\uwave{#1}}} %DIF PREAMBLE
\providecommand{\DIFdel}[1]{{\protect\color{red}\sout{#1}}}                     
%DIF PREAMBLE
%DIF SAFE PREAMBLE %DIF PREAMBLE
\providecommand{\DIFaddbegin}{} %DIF PREAMBLE
\providecommand{\DIFaddend}{} %DIF PREAMBLE
\providecommand{\DIFdelbegin}{} %DIF PREAMBLE
\providecommand{\DIFdelend}{} %DIF PREAMBLE
\providecommand{\DIFmodbegin}{} %DIF PREAMBLE
\providecommand{\DIFmodend}{} %DIF PREAMBLE
%DIF FLOATSAFE PREAMBLE %DIF PREAMBLE
\providecommand{\DIFaddFL}[1]{\DIFadd{#1}} %DIF PREAMBLE
\providecommand{\DIFdelFL}[1]{\DIFdel{#1}} %DIF PREAMBLE
\providecommand{\DIFaddbeginFL}{} %DIF PREAMBLE
\providecommand{\DIFaddendFL}{} %DIF PREAMBLE
\providecommand{\DIFdelbeginFL}{} %DIF PREAMBLE
\providecommand{\DIFdelendFL}{} %DIF PREAMBLE
%DIF LISTINGS PREAMBLE %DIF PREAMBLE
\RequirePackage{listings} %DIF PREAMBLE
\RequirePackage{color} %DIF PREAMBLE
\lstdefinelanguage{DIFcode}{ %DIF PREAMBLE
	%DIF DIFCODE_UNDERLINE %DIF PREAMBLE
	moredelim=[il][\color{red}\sout]{\%DIF\ <\ }, %DIF PREAMBLE
	moredelim=[il][\color{blue}\uwave]{\%DIF\ >\ } %DIF PREAMBLE
} %DIF PREAMBLE
\lstdefinestyle{DIFverbatimstyle}{ %DIF PREAMBLE
	language=DIFcode, %DIF PREAMBLE
	basicstyle=\ttfamily, %DIF PREAMBLE
	columns=fullflexible, %DIF PREAMBLE
	keepspaces=true %DIF PREAMBLE
} %DIF PREAMBLE
\lstnewenvironment{DIFverbatim}{\lstset{style=DIFverbatimstyle}}{} %DIF PREAMBLE
\lstnewenvironment{DIFverbatim*}{\lstset{style=DIFverbatimstyle,showspaces=true}}{} %DIF PREAMBLE
%DIF END PREAMBLE EXTENSION ADDED BY LATEXDIFF

\begin{document}
	
	\maketitle \clearpage
	
	
	\begin{table}[hp] \caption{Revision History} \label{TblRevisionHistory}
		\begin{tabularx}{\textwidth}{llX} \toprule \textbf{Date} &
			\textbf{Developer(s)} & \textbf{Change}\\ \midrule 2020/09/20 & Shayan Mousavi
			& Initial Draft\\ \DIFdelbeginFL %DIFDELCMD <
			
			%DIFDELCMD < %%%
			\DIFdelendFL \DIFaddbeginFL \DIFaddFL{2020/09/24 }& \DIFaddFL{Shayan Mousavi }&
			\DIFaddFL{Draft Revised}\\ \DIFaddendFL \bottomrule \end{tabularx} \end{table}
	
	
	
	\clearpage
	
	
	\newpage
	
	
	\newcommand*\apos{\textsc{\char13}} \section{Introduction}
	
	Surface plasmons are harmonic oscillations of free electrons on the surface of
	materials with high electron density. These oscillations are electromagnetic
	waves following Maxwell's equations and are excited by an incident
	electromagnetic wave (photons or swift electrons). Surface plasmons depend on
	the geometry, dielectric function, and density of charge carriers on the
	surface. Surface plasmons can absorb the energy of the incident light and
	generate hot charge carriers. These generated energetic charge carriers can be
	used for different purposes such as accelerating chemical reactions, designing
	invisibility cloaks, generating local heat for killing cancer cells, and
	biosensing. Being able to simulate the electromagnetic activities on the
	surface gives a great insight into studying and designing novel plasmonic
	systems.
	
	
	\section{Objective}
	
	The purpose of the Surface Plasmon Dynamics Finite Method (SPDFM) script is to
	solve Maxwell's equations applied for surface plasmons in a time domain, for a
	discretized (meshed) anisotropic and varying dielectric environment composed of
	different materials. The \DIFaddbegin \DIFadd{current version of }\DIFaddend
	SPDFM script provides the user with \DIFaddbegin \DIFadd{the plasmon-enhanced
	}\DIFaddend 3D \DIFdelbegin \DIFdel{electromagnetic vector }\DIFdelend
	\DIFaddbegin \DIFadd{electric }\DIFaddend field over the entire
	geometry\DIFaddbegin \DIFadd{, generated by a light source with a frequency
		ranged from mid-infrared to ultraviolet. The time-dependant partial
		differential equations in this script are solved using FEniCS finite element
		toolbox}\DIFaddend .
	
	
	\section{Interest}
	
	Available open-source softwares are either simulating more general optical
	properties and are not adopted for plasmonic properties specifically or
	simulating the plasmonic response with too many restrictions. Among the
	limitations present in the current surface plasmon simulation scripts are
	constrains on the dielectric functions or particle size, and no information on
	the evolution of surface plasmons with time. Although potential equivalent
	licensed softwares might exist, their source code is hidden which forces the
	user to blindly trust them. Interested stakeholders in this project may include
	researchers and industries that are exploring areas related to plasmonic
	physics and devices.
	
	
	
	
	\an{comment}
	
\end{document}

