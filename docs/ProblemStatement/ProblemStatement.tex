\documentclass{article}
\usepackage{tabularx}
\usepackage{booktabs}

\title{CAS 741: Problem Statement\\Surface Plasmon Dynamics Finite Method}

\author{S. Shayan Mousavi M.\\github: @shmouses}

\date{2020/09/20}

%%% Comments

\usepackage{color}

\newif\ifcomments\commentstrue

\ifcomments
\newcommand{\authornote}[3]{\textcolor{#1}{[#3 ---#2]}}
\newcommand{\todo}[1]{\textcolor{red}{[TODO: #1]}}
\else
\newcommand{\authornote}[3]{}
\newcommand{\todo}[1]{}
\fi

\newcommand{\wss}[1]{\authornote{blue}{SS}{#1}} 
\newcommand{\plt}[1]{\authornote{magenta}{TPLT}{#1}} %For explanation of the template
\newcommand{\an}[1]{\authornote{cyan}{Author}{#1}}


\begin{document}

\maketitle
\clearpage


\begin{table}[hp]
\caption{Revision History} \label{TblRevisionHistory}
\begin{tabularx}{\textwidth}{llX}
\toprule
\textbf{Date} & \textbf{Developer(s)} & \textbf{Change}\\
\midrule
2020/09/20 & Shayan Mousavi & Initial Draft\\
\bottomrule
\end{tabularx}
\end{table}


\clearpage
%\tableofcontents
%\clearpage


\newpage


\newcommand*\apos{\textsc{\char13}}
\section*{Introduction}

Surface plasmons are harmonic oscillations of free electrons on the surface of 
materials with high electron density. These oscillations are electromagnetic waves following Maxwell's equations and are excited by an incident electromagnetic wave (photons or swift electrons). Surface plasmons depend on the geometry, dielectric function, and density of charge carriers on the surface.
Surface plasmons can absorb the energy of the incident light and generate hot charge carriers. These generated energetic charge carriers can be used for different purposes such as accelerating chemical reactions, designing invisibility cloaks, generating local heat for killing cancer cells, and biosensing. Being able to simulate the electromagnetic activities on the surface gives a great insight into studying and designing novel plasmonic systems.  
  
  
\section*{Objective}

The purpose of the Surface Plasmon Dynamics Finite Method (SPDFM) script is to solve Maxwell's equations applied for surface plasmons in a time domain, for a discretized (meshed) anisotropic and varying dielectric environment composed of different materials. The SPDFM script provides the user with 3D electromagnetic vector field over the entire geometry. 


 

\section*{Interest}

Interested stakeholders in this project may include researchers, strudents, or 
professors that are exploring areas such as plasmonic physics, plasmon enhanced chemistry and electrochemistry, 

of their systems or effect of these activities in their systems. The processing 
encoded here  focuses on the surface plasmon
The SPD-FE toolbox provide the user with the simulation of the electromagnetic properties of plasmonic structures in time and also it effects on the surrounding environment.


%\an{comment}

\end{document}
