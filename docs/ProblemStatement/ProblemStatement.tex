\documentclass{article} \usepackage{tabularx} \usepackage{booktabs}

\title{CAS 741: Problem Statement\\Surface Plasmon Dynamics Finite Method}

\author{S. Shayan Mousavi M.\\github: @shmouses}

\date{2020/09/24}

%%% Comments

\usepackage{color}

\newif\ifcomments\commentstrue

\ifcomments
\newcommand{\authornote}[3]{\textcolor{#1}{[#3 ---#2]}}
\newcommand{\todo}[1]{\textcolor{red}{[TODO: #1]}}
\else
\newcommand{\authornote}[3]{}
\newcommand{\todo}[1]{}
\fi

\newcommand{\wss}[1]{\authornote{blue}{SS}{#1}} 
\newcommand{\plt}[1]{\authornote{magenta}{TPLT}{#1}} %For explanation of the template
\newcommand{\an}[1]{\authornote{cyan}{Author}{#1}}


\begin{document}
	
	\maketitle \clearpage
	
	
	\begin{table}[hp] \caption{Revision History} \label{TblRevisionHistory}
		\begin{tabularx}{\textwidth}{llX} \toprule \textbf{Date} &
			\textbf{Developer(s)} & \textbf{Change}\\ \midrule 2020/09/20 & Shayan Mousavi
			& Initial Draft\\ 2020/09/24 & Shayan Mousavi & Draft Revised\\ \bottomrule
	\end{tabularx} \end{table}
	
	
	
	\clearpage
	
	
	\newpage
	
	
	\newcommand*\apos{\textsc{\char13}} \section{Introduction}
	
	Surface plasmons are harmonic oscillations of free electrons on the surface of
	materials with high electron density. These oscillations are electromagnetic
	waves following Maxwell's equations and are excited by an incident
	electromagnetic wave (photons or swift electrons). Surface plasmons depend on
	the geometry, dielectric function, and density of charge carriers on the
	surface. Surface plasmons can absorb the energy of the incident light and
	generate hot charge carriers. These generated energetic charge carriers can be
	used for different purposes such as accelerating chemical reactions, designing
	invisibility cloaks, generating local heat for killing cancer cells, and
	biosensing. Being able to simulate the electromagnetic activities on the
	surface gives a great insight into studying and designing novel plasmonic
	systems.
	
	
	\section{Objective}
	
	The purpose of the Surface Plasmon Dynamics Finite Method (SPDFM) script is to
	solve Maxwell's equations applied for surface plasmons, for a
	discretized (meshed) isotropic nonmagnetic dielectric material. The current version
	 of SPDFM script provides the user with the plasmon-induced 3D electric field and current density dynamics over the entire geometry. The
	time-dependant partial differential equations in this script are solved using
	FEniCS finite element toolbox.
	
	
	\section{Interest}
	
	Available open-source softwares are either simulating more general optical
	properties and are not adopted for plasmonic properties specifically or
	simulating the plasmonic response with too many restrictions. Among the
	limitations present in the current surface plasmon simulation scripts are
	constrains on the dielectric functions or particle size, and no information on
	the evolution of surface plasmons with time. Although potential equivalent
	licensed softwares might exist, their source code is hidden which forces the
	user to blindly trust them. Interested stakeholders in this project may include
	researchers and industries that are exploring areas related to plasmonic
	physics and devices.
	
	
	
	
	%\an{comment}
	
\end{document}

